\chapter{Theoretische Grundlagen}\label{chap:t}
Die theoretischen Grundlagen führen die Konzepte ein, die über die ganze Arbeit
hinweg Anwendung finden. Es handelt sich dabei um Zusammenfassungen. Die Theorie
wird auf den Teil reduziert, der für ein grundsätzliches Verständnis der Arbeit
nötig ist. Weitere Informationen werden sind in den referenzierten Quellen
einsehbar. Auch die verwendeten Fachbegriffe werden in diesem Kapitel
eingeführt. 

\section{Machine Learning}\label{chap:t_ml}
Machine Learning ist ein Teilbereich der künstlichen Intelligenz. ``Künstliche
Intelligenz (KI) bezieht sich im Allgemeinen auf jedes menschenähnliche
Verhalten durch eine Maschine oder ein System'' [What is Artificial
Intelligence] Mit Maschinen und Systemen sind in den allermeisten Fällen
Computer, beziehungsweise die steurenden Computerprogramme gemeint. Diese
Computerprogramme bilden ein Modell von menschlichem Verhalten. Machine Learning
Modelle entwickeln (oder erlernen) eine Mustererkennung durch die Analyse von
Daten [What is Machine Learning]. Mustererkennung bedeutet hier, dass der
Algorithmus Zusammenhänge zwischen den analysierten Daten erkennt und auf dieser
Basis Vorhersagen treffen kann. Vereinfacht gesagt, versucht ein Machine
Learning Modell menschliches Urteilsvermögen zu erlernen [is human judgement
necessary].

Ein Beispielproblem für ein Machine Learning Modell ist die Erkennung von
handschriftlichen Zahlen. Ein Computerprogramm soll durch den Input eines Bildes
mit einer handschriftlichen Zahl eine korrekte Beurteilung treffen, um welche
Zahl es sich handelt. Mit anderen Worten soll der Output des Computerprogrammes
der Zahl entsprechen, die auf dem Bild der Eingabe zu sehen ist (siehe autoref{recognition}). Jedes
Computerprogramm, das dieses Problem löst, fällt in den Bereich der künstlichen
Intelligenz. Machine Learning Modelle geben einen Ansatz für die Umsetzung eines
solchen Computerprogrammes.

%todo Bild recognition machine

Machine Learning Modelle, die das Beispielproblem lösen, basieren üblicherweise 
auf Supervised Learning. Das ist ein Teilbereich von Machine Learning, wobei das
Machine Learning Modell aus Rückmeldungen der korrekten Beurteilung, der
Zielvariable, als Reaktion auf ihre eigenen Beurteilungen lernt [datasolut]. Die
Zielvariable muss dabei im Voraus für jeden Datenpunkt in den analysierten Daten
durch einen Menschen festgelegt sein [ML 3.1]. Ausgedrückt durch den Fachbegriff müssen die
Daten labeled sein [2.1 labeled]. Weitere Teilbereiche von Machine Learning sind
Unsupervised Learning und Reinforcement Learning [supervised vs. unsupervised
vs. reinforce]. Beachte \nameref{chap:t_rl} für eine ausgeprägtere Einführung
in Reinforcement Learning.

Machine Learning Modelle sind hauptsächlich in der programmiersprache Python
implementiert [why python]. Dabei werden häufig Tensorflow und Keras verwendet.
Tensorflow ist ein machine Learning Framework [Tensorflow]. Das bedeutet, dass
Tensorflow fertige Funktionen und Algorithmen bereitstellt, die für Machine
Learning Modelle nötig sind. Keras ist ein weiteres Machine Learning Framework,
das selbst mit Tensorflow funktioniert.


\subsection{Funktionsweise eines Machine Learning Modelles}\label{sub:t_ml_func}
Dieser Abschnitt erklärt die Funktionsweise eines Machine Learning Modelles,
basierend auf dem Beispielproblem aus dem letzten Abschnitt (siehe\nameref{chap:t_ml}). 

Bei den Daten, die das Machine Learning Modell analysiert handelt es sich in
diesem Fall um das MNIST Datenset [MNIST]. Dieses Datenset wurde vom NIST
(National Institute of Standards and Technology) in Amerika veröffentlicht und
beinhaltet $70'000$ Bilder von hangeschriebenen Zahlen [NIST EMNIST]. Jedes Bild
hat eine auflösung von $28\times28$ Pixeln (siehe autoref{mnist}).

%todo mnist bild
                
Ein Machine Learning Modell durchläuft ein Training gefolgt von einer Testphase
[Train test split]. Während dem Training erlernt das Modell die Mustererkennung,
um verlässliche Aussagen zu den Daten der Eingabe zu treffen. Die Testphase
misst die Genauigkeit des Modelles, also die Wahrscheinlichkeit, mit der das
Modell die richige Lösung zur Eingabe liefert. Nur in den seltensten Fällen
erreicht diese Genauigkeit $100\%$. Das Modell garantiert somit nicht die
richtige Lösung. Das Machine Learning Modell erlert die Mustererkennung während
dem Training durch die Analyse von Trainingsdaten aus einem Datenset. Das Modell
gibt zu jedem Datenpunkt die Beurteilung, um welche Zahl es sich handelt. Das
Datenset ist labeled (siehe \nameref{chap:t_ml}). Falls die Beurteilung des
Modelles nicht mit der bekannten, korrekten Lösung übereinstimmt, passt sich das
Modell automatisch auf eine bestimmte Weise an. Dadurch soll die Beurteilungen
für zukünftige Datenpunkte genauer werden. Die Testphase misst die Genauigkeit
des Modelles auf Testdaten. Die Testdaten bestehen aus Datenpunkten, die in den
Trainingsdaten nicht enthalten sind.

Zusammengefasst kann Ein Machine Learning Modell Daten Beurteilen und sich
selbst Anpassen, um die Beurteilungen zu verbessern. Künstliche Neuronale Netze
(siehe \nameref{sub:t_ml_nn}) umfassen diese Funktionalität, und finden daher in
Machine Learning Modellen Anwendung.


\subsection{Hyperparameter}\label{sub:t_ml_hyper}
Machine Learning Modelle unfassen verschiedene Hyperparameter. Diese beschreiben
unter anderem wie lange das Training läuft oder wie stark sich das Modell nach
einer falschen Beurteilung anpasst. Diese Hyperparameter beeinflussen das
Lernverhalten des Modelles [Hyperparameter], aber ihr optimaler Wert ist im
Voraus nicht bekannt. 

Hyperparameter können unter anderem durch den Baysian Optimization Algorithmus
optimiert werden [bay Optimization][exploring bayop]. Dieser Algorithmus
versucht, den Output einer Black Box Funktion zu maximieren oder zu minimieren
[bayesopt book s.15]. Eine Black Box ist ein häufig komplexes System, dessen inneren
Vorgänge nicht betrachtet werden [Blackbox Wikipedia]. Bei einer Blackbox
Funktion ist folglich der Input und der Output bekannt, während die Verarbeitung
des Inputs zum Output nicht betrachtet wird (siehe autoref{Blackbox}).

%todo bild Blackbox

Machine Learning Modelle werden häufig als Black Box Funktionen angesehen, da
das Training mit hohem rechnerischen Aufwand verbunden ist, wodurch die genauen
Vorgänge durch einen aussenstehenden Betrachter nicht oder nur schwer erfassbar
sind [how black]. Um ein Machine Learning Modell als eine BlackBox Funktion für
den Baysian Optimization Algorithmus zu verwenden, werden die zu optimierenden
Hyperparameter als Input und eine Zielvariable als Output definiert. Die
Zielvarible des Outputs entspricht dabei einem Wert, der die Leistung des
Modelles widerspiegelt und durch den Algorithmus maximiert werden soll. Ein
Beispiel für die Zielvarible wäre die Genauigkeit des Machine Learning Modelles
(siehe \nameref{sub:t_ml_func}) Die inneren Vorgänge in der BlackBox Funktion
entsprechen in diesem Fall einem Training des Modelles

Der Baysian Optimization Algorithmus kann bis zu 20 Hyperparameter zuverlässig
optimieren [high-dim bay]. Der Algorithmus führt die Blackbox Funktion für eine
bestimmte Anzahl Iterationen mit jeweils verschiedenen Parametern durch. Die
Wahl der Parameter basiert dabei auf Bayes' Theorem [bayesopt book s.7].
Diejenigen Parameter, die den höchsten gefundenen Wert der Zielvariable
auslösen, werden gespeichert.


\subsection{künstliche neuronale Netze}\label{sub:t_ml_nn}
Ein neuronales Netz ist, im biologischen Sinne, "eine beliebige Anzahl Neuronen,
die miteinander Verbunden sind" [Neuro Wikipedia]. Ein Beispiel für ein
neuronales Netz ist das menschliche Gehirn. Künstliche Neuronale Netze
modellieren biologische neuronale Netze in der Form von Programmcode [artificial
neural network]. Diese Arbeit behandelt künstliche neuronale Netze, nicht aber
biologische. Somit handelt es sich bei jedem erwähnten neuronalen Netz, um ein
künstliches neuronales Netz.

Der Grundbaustein eines neuronalen Netzes ist das Neuron. Im Modell stellt das
Neuron ein Objekt dar, das eine beliebiege Anzahl Inputs, aber nur einen Output
hat (siehe autoref{neuron}) [Concept artificial neuron]. Input und Output sind
hierbei rationale Zahlen. Die Ausgabe des Neurons ist im einfachsten Modell, dem
Perzeptron Neuron, grundsätzlich entweder 0 oder 1. Die Ausgabe ist 1, wenn die Summe
der Eingaben einen vorgegebenen Wert, den \emph{Threshold}, überschreitet.
Ansonsten ist die Ausgabe gleich 0. Jede Eingabe hat ein \emph{Gewicht}, das
einer rationalen Zahl entspricht. Vor der Addition der Inputs wird jeder Input
mit seinem Gewicht multipliziert.  Die Grösse des Gewichts bestimmt somit den
Einfluss der zugehörigen Eingabe auf die Ausgabe des Neurons. [neural networks
deep learning][What is Perceptron]

%todo Bild A neuron

Neuronale Netze in Machine Learning Modellen verwenden kompliziertere Neuronen
als das Perzeptron Neuron, wie zum Beispiel das Sigmoid-Neuron. Die Neuronen
unterscheiden sich in ihrer Activation Function und somit im Verhalten ihres
Outputs [activation function]. So nimmt der Output im Sigmoid-Neuron
beispielsweise auch Werte zwischen $0$ und $1$ an in einem stetigen Übergang
zwischen den beiden Grenzen (siehe autoref{sig vs perc}) [sigmoid Neuron].

%todo Bild sigmoid - perceptron funktion

Neuronale Netze sind Verbindungen dieser Neuronen. Dabei dient der Output eines
Neurons als Input in ein anderes Neuron. Der Output eines Neurons kann gleich
für mehrere Neuronen ein Input sein. Die Neuronen sind in \emph{Layers} geordnet
(siehe autoref{layers}). Neuronale Netze haben mindestens eine \emph{Input
Layer} und eine \emph{Output Layer} [neural network deep learning][Neural
network backprop]. Die Input Layer umfasst die Daten, zu dem das neuronale Netz
eine Beurteilung liefert sollte. Im Beispielproblem (siehe \nameref{chap:t_ml})
bestände die Eingabe-Ebene aus $28\times28$ Neuronen, wobei jedes Neuron die
Graustufe (durch einen Wert von $0$ bis $255$) eines Pixels im Bild beschreibt.
Der Input ist in diesem Fall zweidimensional. Die Dimensionen sind allerdings
flexibel. Die Output Layer besteht im Beispiel aus $10$ Neuronen, wobei jedes
Neuron einer Beurteilung entspricht (das fünfte Neuron beschreibt zum Beispiel
die Ziffer Fünf als Beurteilung). Dasjenige Neuron mit dem höchsten Output
entspricht der Beurteilung des neuronalen Netzes.

%todo Bild Layers

Zwischen der Input Layer und der Output Layers kann es weitere \emph{Hidden
Layers} geben [What are hidden layers]. Es gibt verschiedene Arten von Hidden
Layers, die verschidene Funktionen haben. Zwei der meist verwendeten Layers sind
Fully Connected (Dense) Layers und Convolutional Layers [conv vs. fully
connected]. In Fully Connected Layers dient jedes Neuron als Input für jedes
Neuron in der nächsten Layer. In Convolutional Layers trifft das nicht zu.
(siehe autoref{conv vs. full})Die Funktion von Convolutional Layers ist es,
wichtige Merkmale aus dem Input hervorzuheben [convolutional neural network].
Concatenation Layers [concat] sind eine weitere Form von hidden Layers, die zwei
verschiedene Layers als Input haben und diese somit verbinden. Machine Learning
Modelle werden ab mehr als einer Hidden Layer als Deep Learning Modelle
bezeichnet [deep learning].

%todo Bild conv vs full
%file:///C:/Users/robin/Zotero/storage/PAZJPWRZ/convolutional-layers-vs-fully-connected-layers-364f05ab460b.html

Ein Machine Learning Modell passt während dem Training (siehe
\nameref{sub:t_ml_func}) einzelne Gewichte im neuronalen Netz in der
Hoffnung an, dass die Genauigkeit der Beurteilung mit den angepassten Gewichten
grösser ist. Die Genaue Anpassung erfolgt in den meisten Machine Learning
Modellen durch den Backpropagation Algorithmus [Backpropagation][backpropagation].

\section{Reinforcement Learning}\label{chap:t_rl}
Reinforcement Learning bedeutet Lernen durch Interaktion mit einer Umgebung.
[complete guide]. Genauer gesagt soll ein Machine Learning Modell durch
Rückmeldungen und Beobachtungen aus einer Umgebung ein bestimmtes Verhalten
erlernen.

Reinforcement Learning Modelle führen somit die Umgebung ein. Anders als bei
Supervised Learning und Unsupervised Learning (siehe \nameref{chap:t_ml}) sind
die Daten, aus denen das Modell lernen soll, im Voraus nicht bekannt.
Reinforcement Learning Modelle trainieren somit nicht auf der Grundlage eines
Datensets. Das liegt in der Natur der Umgebung, die häufig zu viele verschiedene
Zustände einnehmen kann, als dass diese in einem Datenset gesammelt werden
könnten. Ein Machine Learning Modell kann trotzdem aus einer Umgebung lernen,
indem es selbst mit dieser interagiert und dadurch Erfahrung sammelt. [synopsis]

Als Beispiel kann die echte Welt als eine Umgebung angesehen werden. Der Mensch
wäre in diesem Fall das Reinforcement Learning Modell. Der Mensch lernt die
Eigenschaften seiner Umgebung durch Interaktionen mit dieser kennen.
Beispielsweise lernt ein Mensch die Schwerkraft durch das Hinfallen kennen.
Durch diese Erfahrungen kann der Mensch ein gewissen Verhalten, zum Beispiel das
Laufen, erlernen. Reinforcement Learning Modelle imitieren dieses Lernverhalten.
So verwendet die Robotik häufig Reinforcement Learning, um einen Roboter laufen
zu lassen. Die Umgebung, mit der das Reinforcement Learning Modell lernt, ist
dabei häufig nicht echt, sondern simuliert.

\subsection{Aufbau und Funktionsweise}\label{sub:t_rl_func}
Dieser Abschnitt umfasst eine genauere Erklärung eines Reinforcement Learning
Modelles, in diesem Fall Deep Q-Learning, unter der Verwendung der korrekten
Fachbegriffe.

Ein Reinforcement Learning Modell umfasst eine \emph{Umgebung} und einen
\emph{Agent}. Der Agent ist dasjenige Element in der Umgebung, welches mit
dieser interagiert und daraus lernt [S.B. s.53]. Die Umgebung verändert sich in
Zeitschritten, genannt \emph{Steps}. In jedem Step führt der Agent eine
\emph{Action} aus, die die Umgebung beeinflusst. Die Entscheidung, welche Action
der Agent ausführt, basiert auf einer \emph{Observation} [DQN s.2] der Umgebung.
Die Observation umfasst alle Daten der Umgebung, die für die Entscheidung des
Agents relevant sind. Der Agent trifft seine Entscheidung auf der Basis eines
neuronalen Netzes (siehe \nameref{sub:t_ml_nn}). Der Input in dieses neuronale
Netz ist die Observation der Umgebung und der Output beschreibt die Action, die
der Agent ausführt. Jedes Neuron des Outputs beschreibt eine spezifische Action
des Agents. Der Agent kann somit nur eine feste Anzahl Actions ausführen. Alle
Actions zusammen werden \emph{Action-Space} [S.B s.67] genannt. Jede Action im
Action-Space besitzt einen \emph{Q-Value}, der dem Output des zugehörigen
Neurons entspricht. (siehe autoref{schema1}) [Q-Learning] Die schlussendliche
Entscheidung, welche Action ausgeführt wird, basiert auf der
\emph{Epsilon-Greedy} Strategie [S.B s.34]. Diese Strategie sieht vor, dass die
Entscheidung mit einer Wahrscheinlichkeit von $\epsilon$ auf eine zufällige
Action fällt. Ansonsten fällt die Entscheidung auf diejenige Action mit dem
höchsten Q-Value. Der Agent erkundet die Umgebung durch die zufälligen Actions,
die er teilweise wählt. Der Agent wählt Actions, die er ansonsten nie wählen
würde, und trifft möglicherweise zufällig auf bessere Optionen für zukünftige
Steps [exploration vs. exploitation]. 

%todo bild reinforcement schema 1

Die Umgebung und somit auch der Agent werden durch die Actions des Agenten
beeinflusst. Dieser Einfluss wird durch die \emph{Reward-Function} gemessen. Die
Reward-Function gibt eine rationale Zahl, den \emph{Reward} aus [S.B s.75]. Umso grösser
der Reward, desto positiver ist der Effekt auf die Umgebung und umgekehrt. Ein
positiver Einfluss auf die Umgebung durch eine Action ist so definiert, dass der
Agent durch die Action das gewünschte Verhalten vorzeigt. Die Reward-Function
definiert, welches Verhalten welchen Reward erzielt. Der Q-Value der gewählten
Action wird mit dem Reward (und dem maximalen Q-Value aus den nächsten möglichen
Actions) addiert. Diese Formel nennt sich Bellman-Gleichung [DQN s.3]. Der neue Q-Value          
nimmt somit einen kleineren Wert an, wenn der Reward negativ ist, und einen
grösseren Wert, wenn der Reward positiv ist. Die Gewichte des neuronalen Netzes
werden daraufhin so angepasst, dass der Output für das Neuron, dessen Action
ausgeführt wurde, näher am neu berechneten Q-Value ist (siehe autoref{schema2}).
Der schlussendliche Effekt ist, dass Actions, die einen positiven Reward
auslösen, wahrscheinlicher gewählt werden, und umgekehrt Actions, die einen
negativen Rewards auslösen, unwahrscheinlicher gewählt werden. Der Agent
versucht insgesamt durch seine Actions einen möglichst hohen akkumulierten
Reward zu erzielen [S.B s.57]. Der akkumulierte Reward entsprich der Summe der
Rewards aus jedem Step.

%todo bild reinforcement schema 2

Das Training läuft in \emph{Episodes} [S.B s.14]. Eine Episode umfasst
eine gewisse Anzahl Steps und am Anfang jeder Episode wird die Umgebung in einen
Startzustand zurückgesetzt. Die Resultate eines Steps werden in dem
\emph{Replay-Buffer} gespeichert. Dazu gehören die Observation der Umgebung, die
Action, und der Reward. Der Replay-Buffer enthält Speicherplatz für eine
bestimmte Anzahl Steps. Während dem Training werden zufällige Steps aus dem
Replay-Buffer gewählt, auf die das neuronale Netz trainiert. Das neurnale Netz
trainiert also auf Daten aus der Vergangenheit der Umgebung und des Agents.
Diese Strategie nennt sich Experience Replay [DQN s.5][DQL]. Ausserdem trainiert
das neuronale Netz jeweils mit einem \emph{Batch} an Steps, also mit einer
gewissen Anzahl an Steps gleichzeitig. Der Replay-Buffer und der Batch sichern
zu, dass das neuronale Netz mit einer grossen Vielfalt an Steps trainiert, was
das Lernverhalten stabiler macht als ein chronologisches Training auf einzelne
Steps [training with exp replay, batch].


\section{Verwandte Arbeiten und Themen}\label{chap:t_ver}
Das Nachzeichnen von Strichbildern hängt allgemein mit dem Zeichnen von Bildern
durch einen Computer zusammen. Es gibt verschiedene Ansätze, um einen Computer
zeichnen zu lassen. Ein häufiger Ansatz ist \emph{Stroke-Based Rendering}
[stroke based rendering]. Stroke-Based Rendering ist das Zeichnen von Bilder
durch das Platzieren von Elementen wie Strichen. Beispiele für Arbeiten in
diesem Bereich sind Strokenet [Strokenet] und ``Learning to Paint With
Model-based Deep Reinforcement Learning'' [learning to paint] Andere Ansätze
simulieren die Führung eines Stiftes. (siehe autoref{stroke-based vs Stift}) Ein
Beispiel dafür ist das Programm Doodle-SDQ [Doodle-SDQ]. Doodle-SDQ beschäftigt
sich auch spezifischer mit dem Nachzeichnen von Strichbildern und wird deswegen
im nächsten Abschnitt weiter behandelt.

%todo bild strokebased vs stife

\subsection{Doodle-SDQ}\label{sub:t_ver_dood}
Doodle-SDQ ist ein Computerprogramm, das durch ein Reinforcement Learning
Modell, spezifischer Deep Q-Learning (siehe \nameref{sub:t_rl_func}), erlernt,
Strichbilder aus dem Google QuickDraw Datenset [Quickdraw Image rec]
nachzuzeichnen. Nachfolgend sind die Aspekte von Doodle-SDQ beschrieben, die für
diese Arbeit relevant sind.

Die QuickDraw Bilder, die das Programm nachzeichnen soll, sind zu einer
einheitliche Grösse von $84\times84$ Pixeln verarbeiteitet [Dood s.7]. Der Agent
kann sich auf einer leeren Zeichenfläche von der selben Grösse bewegen und
zeichnen. Die Umgebung umfasst diese Zeichenfläche, den Agent und das
abzuzeichnende Bild.

Der Agent kann sich durch eine Action in jedem Step auf einen beliebigen Pixel
in einem $11\times11$ Feld, in dessen Zentrum er ist, bewegen. Der Agent kann
ausserdem jede dieser Bewegungen im zeichnenden Zustand oder im nicht
zeichnenden Zustand machen. Der Action-Space hat somit insgesamt eine Grösse von
$2\cdot11\cdot11 = 242$ Actions [Dood s.5]. Im zeichnenden Zustand wird ein Strich auf der
Zeichenfläche zwischen der alten und der neuen Position des Agenten gezeichnet.
Der Agent begeht 100 Steps pro Episode. Eine neue Episdoe entspricht dabei einem
neuen Bild, das abgezeichnet werden soll.

Die Obervation der Umgebung, und somit der Input in das neuronale Netz (siehe
\nameref{sub:t_rl_func}), ist in zwei Teile gegliedert: den Global Stream und
den Local Stream. Der Global Stream hat eine Form von $28\times28\times4$. Der
Input ist somit dreidimensional. Die Form kann als 4 aufeindandergestapelte
Bilder angesehen werden, die jeweils eine Grösse von $28\times28$ Pixeln haben.
Dabei beschreibt eine relle Zahl den Wert von jedem Pixel in einem Bild. Das
erste Bild im global Stream ist die Vorlage, die abgezeichnet werden soll. Das
zweite Bild ist die Zeichenfläche im aktuellen Zustand. Das dritte Bild
beschreibt die Position des Agents durch seine relative Entfernung zu jedem
Punkt auf der Zeichenfläche. Das vierte Bild beschreibt, ob der Agent im
zeichnenden Zustand ist oder nicht. Wenn alle Pixel dieses letzten Bildes den
Wert $1$ haben, ist der Agent im zeichnenden Zustand. Wenn umgekehrt alle Pixel
den Wert $0$ haben, ist der Agent nicht im zeichnenden Zustand. Der Local Stream
hat eine Form von $11\times11\times2$. Er ist somit auch dreidimensional und
beschreibt zwei gestapelte Bilder. Das erste Bild umfasst die Vorlage in dem
$11\times11$ Bereich (bezeichnet als Local image patch [dood s.5]), in dem sich
der Agent in einem Schritt bewegen kann. Das zweite Bild beschreibt den selben
Bereich von der Zeichenfläche [Dood s.4 f.]. Der global Stream und der Local
Stream werden durch eine Concatenation Layer (siehe \nameref{sub:t_ml_nn})
zusammengeführt.

Die Reward-Function (siehe \nameref{sub:t_rl_func}) bezieht sich auf die
Differenz der Anzahl der übereinstimmenden Pixel zwischen der Vorlage und
Zeichenfläche zwischen zwei Steps [Dood s.6]. Eine grosse Menge an neu
übereinstimmenden Pixel entsprechen somit einem höheren Reward als wenige neu
übereinsimmende Pixel. Wenn nach einem Step weniger Pixel übereinstimmen als
zuvor, entspricht das einem negativen Reward. 

\section{Git und GitHub}\label{chap:t_git} Git und Github sind weit verbreitete
Hilfsmittel für Software Entwickler. Git ist ein Programm, während GitHub ein
Service ist, der dieses Programm in der Cloud zugänglich macht. GitHub hat
zusätzliche Funktionen, die die Zusammenarbeit zwischen mehreren Entwicklern
erleichtern. Die genaue Funktion und das Zusammenspiel dieser beiden Hilfsmittel
wird nachfolgend erläutert.

\subsection{Git}\label{sub:t_git_git} Git erkennt Veränderungen im Code eines
Projektes und speichert diese Veränderungen in einer neuen Version ab. Die
einzelnen Versionen des Projektes bleiben dabei zu jedem Zeitpunkt abrufbar.
Dieses Konzept nennt sich Version Control [Version Control]. Das Programm wurde
2005 von Linus Torvald entwickelt [Git]. Versionen des Projektes werden manuell
durch einen Commit gespeichert. Es wird empfohlen, nur jeweils ein bestimmtes
Problem oder eine bestimmte Funktion pro Commit anzugehen [Git best practic].
Für grössere Funktionen oder Probleme kann ein Branch erstellt werden. Ein
Branch ermöglicht eine abgekapselte Entwicklung eines Projektes [Git Branch].
Zum Beispiel kann das Projekt in mehreren Branches gleichzeitig und unabhängig
von einander entwickelt werden [Git How Branch]. Eine weit verbreitete
Arbeitsweise und Branch Struktur mit Git ist Gitflow [What is Git Flow]. Gitflow
schlägt grundsätzlich einen Main Branch, einen Develop Branch und verschiedene
Feature Branches vor [Git-flow-workflow]. Professionelle Anwendungen von Gitflow
verwenden ausserdem so genannte Release Branches und Hotfix Branches [Gitflow
release].  Im Main Branch sind offizielle Versionen des Projektes gespeichert,
Im Develop Branch wird das Projekt als ganzes entwickelt, und in jedem Feature
Branch wird eine Funktionalität in das Projektes implementiert (siehe
autoref{Branch Structure} für die genaue Branch Struktur). 



%todo bild branch structure

\subsection{GitHub}\label{sub:t_git_gh}
GitHub wurde 2008 von Chris Wanstrath, PJ Hyett, Scott Chacon und Tom
Preston-Werner entwickelt \cite{noauthor_github_2021}. 2018 wurde das
Unternehmen von Microsoft gekauft. GitHub ist ein Service, der Projekte, die mit
Git verwalten werden, in der Cloud speichert. Dadurch kann ein Projekt überall
und von beliebig vielen Personen entwickelt werden. GitHub betreibt eine
Webseite, über welche der Service verwendet werden kann
\cite{noauthor_github_2021}.

GitHub besitzt verschiedene Hilfsmittel, die die Zusammenarbeit von Entwicklern
weiter vereinfachen. Beispiele dafür sind Issues und Project Boards. Diese
Hilfsmittel ermöglichen Organisation, Strukturierung und Arbeitsteilung. Ein
weiteres Hilfsmittel sind Pull Requests. Eine Pull Reqeust wird dann gestellt,
wenn die Arbeit an einem Branch fertig ist. Durch Pull Requests können die
Entwickler des Projektes die Funktionalität eines Branches überprüfen. Wenn ein
Branch nicht die gewünschte Aufgabe erfüllt, kann die Pull Request abgelehnt
werden. Erst wenn eine Pull Request angonommen wird, kann der Branch wieder in
den Main Branch zurückgeführt werden [Pull Request].

