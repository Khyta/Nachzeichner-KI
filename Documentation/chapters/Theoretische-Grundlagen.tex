\chapter{Theoretische Grundlagen}

\section{Machine Learning}
\label{chap:t_ml}
Teilbereich künstliche Intelligenz (Künstliche Intelligenz ist Maschine, die
menschliche Tätigkeiten nachmacht) Maschine = Computer -> Computerprogramm
Machine Learning ist Überbegriff -> Computerprogramme, die eine Mustererkennung
entwickeln. Durch Analysieren von Daten. KI lernt Vorhersagen

Beispiel Zahlenerkennung: Beispiel für Künstliche Intelligenz Bild einer
handschriftlichen Zahl wird Programm übergeben (Input) Das Programm bestimmt, um
welche Zahl es sich handelt. Programm gibt Zahl aus

Künstliche Intelligenz weil das Erkennen von Zahlen menschlich ist.

Unterschied machine Learning: KI lernt von Grund auf, Zahlen zu Erkennen Anfang
wird allermeistens Falsch sein

KI lernt durch die Analyse von Daten. -> Trainingsdaten. Dabei ist bekannt, was
die richtige Lösung zu den gegebenen Daten ist. So erfährt die KI, wenn sie
falsch liegt. Programm kann sich bezogen auf Erfahrung selbst anpassen.
Anpassungen möglichst so, dass Vorhersagen in Zukunft besser sind.

Nachdem KI mit Trainingsdaten verbessert wurde, sollte diese auch für neue Daten
möglichst genaue Vorhersagen machen.

Nächste Frage: Wie trifft KI entscheidungen, und wie kann sie sich selbst
Anpassen? -> Antwort Neuronale Netze


\subsection*{künstliche neuronale Netze}

Grundbaustein eines neuronalen Netzes sind  Neuronen, die mehrere Eingaben
erhalten. Jede Eingabe wird verschieden gewichtet, wobei das Gewicht mit der
Eingabe (0 oder 1) multipliziert wird. (vereinfachte Erklärung eines
Sigmoid-Neuron)

Wenn Alle Eingaben + Gewichte Addiert werden und den Threshold, einen
vorgegebenen Grenzwert, überschreiten, ist der Output dieses Neurons = 1. -> Das
Neuron feuert

Neuronales Netz = verbindung von künstlichen Neuronen. Künstliche Neuronen
werden von mehreren Eingaben beeinflusst. Die Eingaben haben dabei eine
Unterschiedliche Gewichtung. Das künstliche Neuron hat nur eine Ausgabe zwischen
0 und 1.

In einem neuronalen Netz werden viele Neuronen miteinander verbunden. Der
Eingabe entsprechen dabei die Daten, die dem neuronalen Netz übergeben werden.
Danach dient die Ausgabe von Neuronen als die Eingabe von weiteren Neuronen.
Umso grösser die Ausgabe eines Neurons ist, desto stärker beeinflusst dieses die
Ausgabe der Neuronen, an deren Eingaben es beteiligt ist.

Die Entscheidung des Programms wird auch in verschiedenen Neuronen dargestellt.
Im Beispiel sind es 10 Neuronen, wobei jedes Neuron eine andere Zahl von 0-9
darstellt. Diese Neuronen haben keine Ausgabe mehr, die für das neuronale Netz
relevant ist. Die Entscheidung basiert in diesem Fall lediglich darauf, welches
der 10 Ausgabeneurenen den höchsten Wert enthält. Die Zahl, die dieses Neuron
repräsentiert, wird dann als die Entscheidung des Neuronalen Netzes angesehen

Der Lernprozess findet durch eine anpassung von Einzelnen Gewichten der
Eingaben, wenn die KI eine falsche Entscheidung trifft. Dadurch soll die
Entscheidung so angepasst werden, dass sie im nächsten Fall besser ausfällt




\section{Reinforcement Learning}
\label{chap:t_rl}
Verschiedene Arten


\section{Verwandte Arbeiten und Themen}
\label{chap:t_verwandt}

Ein verwandtes Thema dieser Arbeit ist die Robotik. Ein Roboter ist laut der
Definition eine `Apparatur, die bestimmte Funktionen eines Menschen ausführen
kann' (Duden). Das Nachzeichnen von Strichbildern ist ebenfalls eine menschliche
Tätigkeit. Diese Arbeit untersucht allerdings nur das Computerprogramm, welches
diese Tätigkeit verrichten kann. Die Apparatur, die von diesem Programm
gesteuert werden könnte, wird also nicht erbaut.

Es gibt verschiedene Ansätze, um ein Computerprogramm die menschliche Tätigkeit
des Nachzeichnens verrichten zu lassen. Ein häufiger Ansatz ist "Stroke-Based
Rendering", wobei Bilder durch das Platzieren von Elementen wie Strichen
gezeichnet werden. Beispiele für Arbeiten, die diesen Ansatz verwenden sind\dots
Stroke-Based Rendering unterscheidet sich von menschlichem Zeichnen dadurch,
dass kein Stift geführt wird. Stattdessen können die Elemente zu jedem Zeitpunkt
an einer willkürlichen Position auf der Zeichenfläche platziert werden.

Andere Ansätze simulieren die Führung eines Stiftes. Damit ist gemeint, dass das
Computerprogramm nicht zu jedem Zeitpunkt an jedem Ort Zeichnen kann.
Stattdessen muss das Programm einen virtuellen Stift bewegen und kann nur gerade
dort zeichnen, wo sich der Stift befindet. Das ist eine Einschränkung, die auch
auf menschliches Zeichnen zutrifft. Ein Beipiel für diese Art des Zeichnens ist
Doodle-SDQ. Das Computerprogramm dieser Arbeit basiert auf dem Programm von
Doodle-SDQ 

\subsection*{Doodle-SDQ}
Doodle-SDQ ist ein Programm, das durch Deep Q Learning erlernt hat, Strichbilder
aus dem Google Quick-Draw Datenset nachzuzeichnen. Ist in der Freiheit des
Zeichnens eingeschränk\indent 

Architektur: Agent = Stift. Umgebung = Zeichenfläche. Speichert Position von
Agent, das zu Nachzeichnende Bild, das was bis jezt gezeichnet wurde und ob
Stift gerade vom Blatt gehoben ist. Die Belohnung/Bestrafung ist der Grad der
Ähnlichkeit zwischen Bildern. Die Unmittelbare Umgebung des Agenten, in dem er
sich in einem Schritt bewegen kann, wird in das neuronale Netz noch ein weiteres
Mal eingegeben, wodurch darauf ein Fokus gelegt wird. Ausgabe des neuronalen
Netz ist eine der Aktionen die der Agent ausführen kann
R

\section{Git und GitHub}
\label{chap:t_git}



