\chapter{Einleitung}
Der Computer ist ein Werkzeug. Er kann den Menschen Arbeit abnehmen. Um
komplizierte Aufgaben zu übernehmen, muss sich der Computer an menschliches
Verhalten, menschliches Urteilsvermögen und menschliche Intelligenz annähern.
Mit anderen Worten braucht der Computer, oder das steuernde Computerprogramm
eine künstliche Intelligenz. Es ist schwierig, Intelligenz in einem
Computerprogramm umzusetzen. Der fähigste und am weitesten verbreitete Ansatz
liefert Machine Learning. 

Diese Arbeit selbst ist eine Untersuchung Im Bereich Machine Learning.
Spezifischer ist die Arbeit im Bereich Reinforcement Learning, einem Teilgebiet
von Machine Learning.
%Das Wort künstliche Intelligenz kommt oft vor. Abgekürzt zu KI
% BUG: künstliche Intelligenz abzürken -> KI

Die Fragestellung der Untersuchung lautet: Kann eine künstliche Intelligenz
lernen, Strichbilder nachzuzeichnen, sodass sie durch ein automatisches System
richtig erkannt werden können?

Für ein gegebenes Strichbild soll die künstliche Intelligenz erlernen, ein
möglichst gleiches Bild daneben zeichnen können. Die Frage ist, ob die
künstliche Intelligenz das Nachzeichnen genug gut lernen kann, damit die
Zeichnung von einem automatischen System richtig erkannt wird. Richtig erkannt
heisst in diesem Fall vereinfacht, dass eine zweite künstliche Intelligenz in
der Zeichnung das selbe Motiv wie in der Vorlage erkennt. Wenn das zutrifft,
zeichnet die künstliche Intelligenz erfolgreich nach. Es gibt allerdings
weitere Kriterien, die den Erfolg der künstlichen Intelligenz beim
Nachzeichnen beurteilen.

Nachzeichnen ist eine menschliche Tätigkeit. Menschen führen beim Zeichnen durch
gewisse Handbewegungen einen Stift, wodurch das Nachzeichnen mit physischen
Einschränkungen verbunden ist. Der Stift kann sich nicht teleportieren, sondern
sich nur mit einer bestimmten Geschwindigkeit fortbewegen. Die künstliche
Intelligenz soll das Nachzeichnen mit ähnlichen physischen Einschränkungen
erlernen. Mit anderen Worten soll die künstliche Intelligenz lernen, einen
Stift zu führen.  Die physischen Einschränkungen der künstlichen Intelligenz
sind dabei jedoch simuliert und im Vergleich zu der echten Welt vereinfacht. 

Die künstliche Intelligenz soll das Nachzeichnen von Strichbildern allgemein
erlernen. Strichbilder können Zahlen, Buchstaben, Formen, Symbole und
allgemeine Kritzeleien sein. Die künstliche Intelligenz kann nicht mit allen
Arten von Strichbildern trainieren, weil die Varianz zu gross ist. Die Frage
ist also, wie sich die künstliche Intelligenz beim Nachzeichnen von
Strichbildern verhält, die im Training nicht enthalten sind.

Die vorangehenden Überlegungen sind in einer Sammlung an Unterfragen, die in
dieser Arbeit beantwortet werden, enthalten. Die Unterfragen lauten:
\begin{itemize}
    \item Wie kann die Architektur einer künstlichen Intelligenz aussehen, die das Nachzeichnen erlernt?
    \item Nach welchen Kriterien lässt sich die Leistung der künstlichen Intelligenz in dieser Aufgabe beurteilen?
    \item Wie lässt sich die Leistung der Künstlichen Intelligenz in dieser Aufgabe verbessern?
    \item Wie ändert sich die Leistung der künstlichen Intelligenz für Strichbilder, die im Training nicht enthalten sind?
    \item Welche Einflüsse haben physische Einschränkungen auf die Leistungs der künstlichen Intelligenz?
    \item Wie und in wiefern lässt sich die künstlichen Intelligenz mit menschlichem Zeichnen vergleichen?
\end{itemize}


