\chapter{Resultate}
Resultate werden in Testumgebung dargestellt 
Trainingsverhalten irrelevant 
(vielleicht nur während Methode trainingsplot zeigen)

\section{Methode zur Darstellung}
- Testumgebung
    - 6000 Test Bilder
    Gleiche Bedingungen wie am Ende des Trainings
- Art der Darstellung
    - Zu jeder Version Daten aus Tests (Vergleichstabelle)
    - Zu jeder Version Beispielbilder
    - zu jeder Version ein gif

\subsection*{Kriterien der Auswertung}
Nur Objektive Kriterien (siehe Methode Evaluierung)
    - relative Pixelähnlichkeit (Prozent)
    - MNIST erkennung (Prozent)
    
    (vielleicht:
      - Durchschnittliche Geschwindigkeit (Unterschied Base, Physik)
      - Geschwindigkeit bis Endprodukt (z.B sobald 50 \% und MNIST erkennung)
    )

    
\subsection*{Was wird Ausgewertet}
Versionen:
    - Base Version
    - Physik Version
    (- Base Verision MNIST Training)
    (- Physik Version MNIST Training)

    Base Version + Speed 
    Physik Version + Speed 
    
    (Base Version + MNIST-Speed)
    (Physik Version + MNIST-Speed)


\section{Darstellung}
(Wenig Text, viele Tabellen, so wie bei Art der Darstellung erklärt)








