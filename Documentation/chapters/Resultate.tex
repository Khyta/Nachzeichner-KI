\chapter{Resultate}
Die Resultate bestehen aus drei Tabellen. Jede Tabelle beschreibt die Leistung
der acht Versionen {siehe methode Variationen}, bezogen auf die drei definierten
Kriterien {siehe methode Evaluation der Leistung}. Die Daten in den Tabellen
stammen aus den Tests der künstlichen Intelligenz {siehe methode. Auswertung}
Der unterschied in den Tabellen liegt im Datenset, mit denen die Versionen der
künstlichen Intelligenz jeweils getestet sind. 
Eine Sammlung von gezeichneten Strichbildern ergänzt die Resultate. Die
Strichbilder sind dabei jeweils in Paaren angeordnet. Das linke Bild im Paar
zeigt die Vorlage aus dem Datenset und das rechte Bild zeigt die nachgezeichnete
Variante von der künstlichen Intelligenz. Die Zeichnungen, die in der Sammlung
vertreten sind, sind zufällig ausgewählt aus dem Test der jeweiligen Version der
künstlichen Intelligenz. Die Bilder haben einen Farbverlauf, der den zeitliche
Verlauf des Zeichnens dargestellt. Die Helligkeit eines Striches ist
proportional zu dem Step, in dem dieser gezeichnet wird. Das bedeutet, dass
dunklere Striche früher gezeichnet werden als hellere Striche. Bewegungen des
Agenten, in denen dieser nicht zeichnet, sind in den Bildern nicht erkennbar.


\section{Tabellen}


\section{Bildersammlung}
