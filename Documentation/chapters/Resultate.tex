\chapter{Resultate}
Bei dem Resultat der Methode handelt es sich um die Leistung der
Computerprogramme bezogen auf die ausgewählten Kriterien (siehe chap evaluierung  % TODO: Ref setzen
der Leistung). Mit den Computerprogrammen sind dabei die unterschiedlichen, auf
dem Grundprogramm basierenden Versionen gemeint, die auf die verschiedenen
Kriterien ausgelegt sind. Mit den Daten über die Leistung der Computerprogramme
ist eine genauere Untersuchung der Fragestellung möglich.

\section{Methode zur Auswertung}
Die Leistung der Computerprogramme wird in einer Testumgebung ausgewertet. In
dieser Testumgebung wird die Leistung eines trainierten Agenten überprüft. Die
Testumgebung unterscheidet sich kaum von der Trainingsumgebung. Die einzigen
Unterschiede sind, dass der Agent in keinem Fall mehr eine zufälligen Aktion
trifft ($epsilon = \epsilon = 0$), und die Bilder, die als Vorlage dienen. In
der Testumgebung werden dem Computerprogramm $1000$ Bilder von handgeschriebenen
Zahlen aus dem Testset übergeben. Das Testset besteht aus $6000$ Bildern. Diese
Bilder sind im Trainingsset nicht vertreten, was bedeutet, dass der Agent für
ihn unbekannte Bilder von Zahlen nachzeichnen muss. Am Ende jeder Zeichnung
werden die relevanten Kriterien ausgewertet. Am Ende des Testes, sobald alle
Zeichnungen fertig sind, wird für jedes Kriterium der Durchschnitt aus allen
1000 Zeichnungen berechnet. Diese Werte repräsentieren die Leistung des
Computerprogrammes.

\subsection*{Auszuwertende Programme und Kriterien}
Eine Auswertung durch die Testumgebung ist nur für messbare Kriterien und
teilweise messbare Kriterien möglich, weil diese durch einen berechenbaren
Zahlenwert repräsentiert sind (siehe chap evaluierung). Am Ende jeder Zeichnung  % TODO: Ref setzen
werden folgende Kriterien berechnet:
\begin{itemize}
    \item prozentuale Genauigkeit
    \item Erkennbarkeit
    \item Geschwindigkeit
\end{itemize}
Die prozentuale Genauigkeit ist eine Prozentangabe für den Anteil der
getroffenen weissen Pixel zwischen der Vorlage und der fertigen Zeichnung. Die
Erkennbarkeit ist durch eine Eins repräsentiert, wenn in der Zeichnung die
korrekte Zahl von einer Schrifterkennungssoftware erkannt wird. Ansonsten ist
die Erkennbarkeit durch eine Null repräsentiert. Der Durchschnitt aus diesem
Kriterium jeder Zeichnung entspricht einer Angabe des Anteils der korrekt
erkannten Zahlen. Dieser Anteil wird in Prozent angegeben. Die Geschwindigkeit
ist durch die die Anzahl Schritte bis zur Fertigstellung definiert. Eine
geringere Zahl entspricht hier einer besseren Leistung.

Für folgende Versionen wird die Leistung in den drei Kriterien ausgewertet.
\begin{itemize}
    \item Grundprogramm (Kriterium der prozentualen Genauigkeit)
    \item 2D-Physik-Programm
    \item 3D-Physik-Programm
    \item Grundprogramm + Training auf Erkennbarkeit
    \item 2D-Physikprogramm + Training auf Erkennbarkeit
    \item 3D-Physikphysikprogram + Training auf Erkennbarkeit
    \item Grundprogramm + Training auf Geschwindigkeit
    \item 2D-Physikprogramm + Training auf Geschwindigkeit
    \item 3D-Physikphysikprogram + Training auf Geschwindigkeit
\end{itemize}
% TODO: Überall refs setzen






% Testumgebung. Die gleiche Umgebung wie bei dem Training des Agenten. 

% - Testumgebung
%     - 6000 Test Bilder
%     Gleiche Bedingungen wie am Ende des Trainings
% - Art der Darstellung
%     - Zu jeder Version Daten aus Tests (Vergleichstabelle)
%     - Zu jeder Version Beispielbilder
%     - zu jeder Version ein gif

    
\subsection*{Was wird Ausgewertet}
Versionen:
    - Base Version
    - Physik Version
    (- Base Verision MNIST Training)
    (- Physik Version MNIST Training)

    Base Version + Speed 
    Physik Version + Speed 
    
    (Base Version + MNIST-Speed)
    (Physik Version + MNIST-Speed)


\section{Darstellung}
(Wenig Text, viele Tabellen, so wie bei Art der Darstellung erklärt)








