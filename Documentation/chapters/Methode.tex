\chapter{Methode}
Die Methode, um die Fragestellung zu beantworten besteht aus zwei Teilen. Diese
Teile wiederholen sich im Verlaufe der Arbeit und sind nicht chronologisch
geordnet. Im einen Teil werden die Kriterien gefunden, nach denen die Leistung
des Programms evaluiert werden kann. Im anderen Teil wird das Computerprogramm
realisiert, das bezogen auf die Kriterien die gewünschte Leistung erbringt. Die
Antworten auf die Unterfragen, zum Beispiel wie ein solches Programm aussehen
kann, laufen in den Prozess mit ein. 

Es gibt verschiedene Kriterien für die Leistung eines Computerprogramms, das
Menschen beim Zeichnen einer Zahl nachahmt. Einige Kriterien sind untereinander
kombinierbar, andere schliessen sich gegenseitig aus. So entstehen verschiedene
Versionen des Programmes, deren Leistung auf verschiedenen Kriterien beruht.


\section{Grundprogramm}
Es ist Teil der Methode, ein Computerprogramm zu entwickeln, das bezogen auf ein
Kriterium ein möglichst gutes Ergebnis erzielt. Reinforcement Learning
Algortihmen stellen diese Funktion bereit. Auch wenn sich das Kriterium ändert,
wonach sich das Computerprogramm richtet, kann der Reinforcement Learning
Algorithmus dahinter grösstenteils gleich bleiben. Der passende Reinforcement
Learning Algorithmus ist somit ein geeignetes Grundprogramm, worauf alle
Erweiterungen stützen.

Das Grundprogramm basiert auf dem Reinforcement Algorithmus aus dem Programm
Doodle-SDQ von Tao Zhou et Al (2018). Das Grundprogramm in dieser Arbeit
übernimmt die Architektur, bezogen auf die Form der Eingabe, der Ausgabe und den
versteckten Ebenen des Neuronalen Netz. 

\subsection*{Doodle-SDQ als Basis}
Bei der Umgebung hantelt es sich, wie bei Doodle-SDQ, um eine Zeichenfläche,
worauf sich der Agent frei bewegen kann. 

Die Zahlen, die nachgezeichnet werden sollen, stammen aus dem MNIST Datenset und
haben somit eine Grösse von 28x28 Pixeln. Die Fläche, worauf sich der Agent
bewegen kann, hat somit auch eine Grösse von 28x28 Pixeln. die "globale" Eingabe
in das Neuronale Netz ändert sich bis auf diese neue Grösse der Bilder nicht.

die Bilder, wie auch die Zeichenfläche, haben die Datenstruktur einer Bitmap. Es
handelt sich also um eine 28x28 Matrix, wobei jedes Element eine Null oder eine
Eins ist. Eine Null repräsentiert einen schwarzen Pixel an dieser Stelle im Bild
und eine Eins einen weissen Pixel.

Die Position des Agenten hat die Datenstruktur einer Liste mit zwei Elementen,
die die horizontalen und vertikalen Koordinaten auf der Zeichenfläche
repräsentieren

Die lokale Eingabe, also das nahe Umfeld um den Agenten schrumpft von 11x11
Pixeln auf 5x5 Pixeln. Somit schrumpft gleichzeitig der Actionspace des Agenten
von 2*11*11 = 242 Aktionen auf 2*5*5 50 Aktionen. Das bedeutet für den Agenten,
dass er sich pro Schritt um maximal zwei Pixel von seiner Position pro Schritt
bewegen kann. Diese bewegung kann der Agent wie in Doodle-SDQ entweder
zeichnend oder nicht zeichnend ausführen.

Falls der Agent die Aktion zeichnend ausführt, zieht das Programm einen Strich
zwischen der alten und der neuen Position. Mit anderen Worten werden alle Pixel
in der Zeichenfläche zwischen den beiden Positionen weiss gemacht. der Strich
hat eine festgelegte Breite von 3 Pixeln.

Am Anfang jeder Episode, also bei jeder neuen Zahl, die gezeichnet werden soll,
startet der Agent in einer zufälligen Position im nicht zeichnenden Zustand.

Aktionen des Agenten, die ihn über die vorgegebene Zeichenfläche hinaus
positionieren würden, sind nicht zulässig. Diese Aktionen können vom Agenten
nicht gewählt werden und ihr optimaler Q-Wert ist als 0 definiert. Das hat zur
Folge, dass nach der Trainingsphase die allermeisten unzulässigen Aktionen einen
tiefen Q-Wert haben. Das senkt die Wahrscheinlichkeit, dass der Agent versucht,
eine unzulässige Aktion zu wählen.

der Reward wird wie in Doodle-SDQ aus den neu übereinstimmenden Pixeln zwischen
der Vorlage und der Zeichenfläche pro Schritt definiert. Der Reward besteht
allerdings aus einer prozentualen Übereinstimmung der Pixel anstelle eines
absoluten Rewards pro Pixel. Auf die Eigenschaften und die Unterschiede dieser
beiden Reward-Funktionen wird im nächsten Abschnitt eingegangen (siehe
Evaluierung). Anders als bei Doodle-SDQ bekommt der Agent keine negativen
Rewards für langsame Bewegungen. 

Alle Aspekte des Grundprogrammes sind leicht veränderbar, um es für
Erweiterungen flexibel zu halten.

% Wird Rare-Exploration verwendet?  Wenn ja: Todo
%- Umgebung (drawline Funktion, Bilder nur als Bitmaps, Translation der Aktion, Render, Anfang in random Position)
%- Agent (Eingabe 28x28, Ausgabe 7x7, rareexploration, Illegale Aktionen gegen 0 gehen lassen, (Relativer Reward = Kriterium zur Leistung der KI. prozentuale Übereinstimmung der Pixel. -> Genauer beschrieben im Kapitel Evaluierung der Leistung)
%- Unterschiede (Keine Stroke Demonstration, kein Training auf Geschwindigkeit)

\subsection*{Präparierung der Daten und Training}
Das in dieser Arbeit verwendete MNIST Datenset besteht aus 42000 Bildern von
handgeschriebenen Zahlen zwischen Null und Neun. Die Bilder im Datenset sind als
Bitmap dargestellt, wobei jedes Element (jeder Pixel) einen Wert zwischen 0 und
255 annimmt. Die Zahl räpresentiert einen Punkt auf dem Spektrum von Grautönen,
wobei 0 Schwarz ist und 255 Weiss. Diese Graustufen werden entfernt. Jeder Pixel
mit einem Wert über 0 übernimmt den Wert 1. So stimmen die Bilder mit den
Zeichnungen, die der Agent produzieren kann, überein.

Die Trainingsdaten bestehen aus 36000 der 42000 Bilder im Datenset. Die
restlichen 6000 Bilder sind für die Testphase aufgehoben. Das Grundprogramm
trainiert mit 4000 Bildern, von denen jede Zahl von Null bis Neun 400 Bilder
ausmacht. Die restlichen Bilder in den Trainingsdaten sind für mögliche
Erweiterungen aufgehoben. Der Agent zeichnet jedes der 4000 Bilder insgesamt
drei mal. Mit anderen Worten Läuft die Trainingsphase für 3 Epochen mit jeweils
4000 Episoden. Der Agent macht 64 Schritte pro Episode. Er kann sich also pro
Zeichnung 64 mal bewegen. Insgesamt trainiert der Algorithmus somit auf der
Basis von 3*4000*64 = 780'000 Schritten. Der Replay Buffer speichert die
Schritte von 700 Episoden. Somit speichert er 700*64 = ... Schritte. Das
Training (des Neuronalen Netz) findet in jedem vierten Schritt statt mit einem
Batch von 64 zufällig gewählten Schritten aus dem replay buffer. 


\section{Evaluierung der Leistung}
In diesem Abschnitt sind die Kriterien definiert, nach denen die Leistung des
Computerprogramms bestimmt werden kann. Diese Kriterien sind relevant für die
Beantwortung der Fragestellung dieser Untersuchung. Es gibt zwei Kategorien von
Kriterien: messbare Kriterien und nicht messbare Kriterien. Zwischen den
Kategorien gibt es weitere Kriterien, die teilweise messbar sind.

\subsection*{Messbare Kriterien}
Messbare Kriterien sind in diesem Fall objektiv. Das heisst sie sind nicht
abhängig von der Wahrnehmung eines Betrachters. Messbare Kriterien sind durch
berechenbare Zahlenwerte definiert und stützen auf Daten, die direkt aus dem
Computerprogramm hervorgehen. Aus diesen Gründen eignen sich messbare Kriterien
als Reward-Funktionen für einen Reinforcement Learning Algorithmus. Eine
Reward-Funktion berechnet allerdings nur den Unterschied im Wert des Kriteriums
zwischen zwei Schritten anstelle des eigentlichen Wertes (siehe Th RL)

\emph{Die Anzahl sich unterscheidender Pixel zwischen der Vorlage und der Zeichnung}
ist ein messbares Kriterium. In anderen Worten misst dieses Kriterium die
Genauigkeit des nachgezeichneten Bildes. Allerdings wird nicht die Anzahl der
übereinstimmenden Pixel bestimmt, sondern der sich unterscheidenden. Das
bedeutet, dass ein tieferer Wert dieses Kriteriums einer besseren Leistung
entspricht. Der Vorteil davon liegt darin, dass die Pixel, die bereits zwischen
einer leeren Zeichenfläche und der Vorlage übereinstimmen, nicht mitgezählt
werden.

Das Grundprogramm dieser Arbeit verwendet eine abgeänderte Form dieses
Kriteriums: \emph{Die Prozentuale Übereinstimmung der weissen Pixel zwischen der Vorlage
und der Zeichnung}. Wie im letzten Kriterium werden die sich unterscheidenden
Pixel gezählt. Die Anzahl Pixel, die sich zwischen der
Vorlage und einer leeren Zeichenfläche unterscheiden, ist relevant. Diese Zahl
$S_(max)$ definiert die maximale Anzahl Pixel, die richtig gezeichnet werden
können. Der Wert $K(t)$ des Kriteriums zu einem bestimmten Schritt berechnet
sich nun aus folgender Formel: 
$$ K(t) = 1 - \frac{S(t)}{S_max} $$ Dabei ist $S(t)$ die Anzahl sich
underscheidender Pixel zu dem Schritt an dem das Kriterium ausgewertet werden
soll. Bei diesem Wert handelt es sich also um die prozentuale (als Dezimalzahl)
Übereinstimmung der Weissen Pixel. Dieser Wert kann auch negativ sein, wenn
weisse Pixel an Stellen gezeichnet werden, an denen in der Vorlage keine sind.
Dieses Kriterium hat den Vorteil, dass die Leistung bei Vorlagen, die nur wenige
weisse Pixel haben (wo zum Beispiel die Zahl klein geschrieben ist), nicht
kleiner eingeschätzt wird, als bei Vorlagen mit vielen weissen Pixeln.
    

\subsection*{nicht Messbare Leistungen}
Nicht messbare Leistungen sind subjektiv. Sie sind abhängig von der Wahrnehmung eines Betrachters.





Ziel: Eine KI. Ein Programm, das Menschen nachahmt. 
Kann Menschen besser oder schlechter nachahmen -> Leistung
Schwer messbar. Subjektiv. Auf Beobachtungen gestützt

    (Probleme mit Grundprogramm?)
    
    Wichtigstes Prinzip: Schwungvoll. -> Hohe Geschwindigkeit, wenige bis keine Abbrüche
        - Implementierung von Physik. 
        - Erkennbarkeit über Genauigkeit


\section{Physikalische Umgebung}
% Art von Physik: Kinematik. Addierung von Geschwindigkeit 2 Dimensionale
% Kinematik. Stift bewegt sich nur auf Blatt -> einziges 3D Element ist hebung und
% senkung von Stift. Wie bis Anhin in 2 States geregelt. (Also kein Druck -> Entfernung von menschlichem Zeichnen)
Bei der physikalischen Umgebung für die Untersuchung handelt es sich um ein sehr
simples inertial System. Also ein System in dem die Gesetze der Kinematik
gelten. Das Ziel dieser Veränderung ist es die Anzahl möglicher Aktionen des
Agenten zu reduzieren, wodurch es einfacher werden soll diese zu erlernen.
Zusätzlich soll die simulierte Umgebung auch einer reellen Situation näher
kommen und somit sich mehr an das menschliche Zeichnen annähern.

Die Umgebung nimmt Kraftvektoren entgegen und berechnet darauf hin die neue
Stiftposition auf der Zeichenebene. Der Kraftvektor wird gemäss der Gesetze der
Kinematik zu der schon bestehenden Geschwindigkeit hinzugefügt. Jeder
Zeitschritt des Agenten repräsentiert in der Physik die vergangene Zeit $t=1$.
($\frac{\vec{F}}{m}\cdot t=\vec{v}$). Pro Zeitschritt wird die Geschwindigkeit
auch kleiner durch eine simulierte Reibung ($m\cdot g \cdot \mu = F_R$). Für die
Position wird immer die Geschwindigkeit zu der jetzigen Position addiert.

Wie aus den Formeln herauszulesen ist, besitzt der Stift und auch die Umgebung
physikalische Eigenschaften ($m$, $g$, $\mu$), welche für ein optimales Ergebnis
später optimiert werden.

Der Agent hat in der Umgebung die Möglichkeit haben seinen Stift mit
Kraftvektoren zu beschleunigen. Nicht nur eine Beschleunigung ist möglich,
sondern das Gleiten mit der restlichen Geschwindigkeit ist für den Agenten
möglich.

Um das menschlichte Zeichnen noch näher zu bringen kann sich der Agent
zusätzlich entscheiden, wie stark der Stift auf die Zeichenfläche drückt. Das
bewirkt, dass um den Stift herum Pixel zusätzlich bemalt werden. Zu der Stärker
kommt auch dazu, dass der Agent sich entscheiden kann, gar nicht auf das Papier
zu drücken, sondern sich ganz vom Papier zu heben und nicht zu zeichnen.

In der Realität müsste der Druck auch mit einer Beschleunigung gelöst werden,
was zur Vereinfachung weggelassen wird.

\subsection*{Training auf physikalische Umstände}
% Ausgabe in Form von Kraftvektoren in Kreis angeordnet. Kraftvektoren
% beeinflussen Geschwindigkeit des Agenten. Position in jedem Schritt wird über
% Geschwindigkeit bestimmt. Durchgehende Reibung wirkt auf Agenten
% -> Physik beruht ausnahmslos auf Dynamik und Kinematik. Vereinfachung durch Annahmen

% Geschwindigkeit Als Eingabe (-> Absolute Werte, oder verschiebung des Patch)
% Problem von zu hoher Geschwindigkeit (-> Lösung durch negative Belohnung)


Da die Aktiondes Agenten nur ein einfacher Integer ist, hat jeder Kraftvektor
einen eigenen Index bekommen. Es gibt insgesamt $20 + 1 (\text{gar nicht     %Der Action-space besteht aus 21 Aktionen im zeichnenden Zustand und die selben 21 Aktionen im nicht zeichnenden Zustand.  
bewegen})$ Richtungen in die der Agent sich bewegen kann. Der Stift wird somit %Die 21 Aktionen bestehen aus Kraftvektoren die in verschiedene Richtunge um den Agent zeigen.
immer in eine dieser Richtungen, um den Betrag $1$ beschleunigt. Mit den
Optionen des Drucks kommt die Formel für die Anzahl Aktionsmöglichkeiten heraus:
$21\cdot(\underbrace{1}_{\text{nicht zeichnen}} + \underbrace{s}_{\text{maximale
Stärke des Zeichnens}})$. So sind die Möglichkeiten der Aktionen des Agenten
viel stärker begrenzt als in der ursprünglichen Variante. % Von den 50 Aktionen der Basisversion auf 42 gesenkt

Es stellte sich zu beginn des Training heraus, dass der Agent das Konzept der
Geschwindigkeit noch nicht verstanden hat. Um dieses Problem zu lösen wurde die % Die Geschwindigkeit des Agenten zu einem gegebenen Zeitpunkt ist relevant für die Entscheidung der nächsten Aktion.
Geschwindigkeit als Kontext zur Observation hinzugefügt und somit als Eingabe   % Deswegen müssen Informationen über die Geschwindigkeit 
des Netzes verwendet. Dies brachte bessere Ergebnisse, allerdings gelang es noch
bessere Ergebnisse zu erziehlen, in dem man die Geschwindigkeit nicht roh als
Zahlen mitgibt, sondern den lokalen Patch in die Richtung der Geschwindigkeit,
also der nächsten Position des Agenten verschiebt.

Ein weiteres Problem während dem Training war, dass der Agent viel zu hohe
Geschwindigkeiten angenommen hat und somit oft aus der Zeichenebene verschwinden
wollte. Dieses Problem wurde durch die Reward Function gelöst, sodass er bei
solchem Verhalten bestraft wird.

% TODO: Übersetzung Lokalen Patch nachschauen
% TODO: War negativer Reward wirklich die Lösung?

\subsection*{Parameteroptimizerung}
Nach einigem Trainieren zeigt sich, dass der Agent am meisten Erfolg mit einer
maximalen Zeichenstärke von $1$ hat. Zur Optimierung der physikalischen
Parameter wird gleich, wie bei der Hyperparameteroptimierung (siehe % TODO: Ist das wird hier so gut gelöst?
\nameref{chap:Hyperparameter Optimierung}) der Baysian Algorithmus verwendet um
das bestmöglichste Ergebnis zu erzielen.

% TODO: Ist das Überhaupt eine gute Idee, weil wir so die Aufgabenstellung des menschlichen Zeichnens verfremden?


\section{Schrifterkennung als Kriterium}
Möglichkeit einer neuen messbaren Leistung
-> Annäherung an Menschliches Verhalten weil Erkennbarkeit > Genauigkeit


\subsection*{Training auf bessere Erkennbarkeit}
- Erkennung von MNIST Netz als Reward. 
verschiedene Varianten (MNIST Training only, Pixelreward mit MNIST am Schluss, Pixelreward - MNIST reward Verlauf, MNIST zu wenigen Schritten)


\section{Hyperparameteroptimierung}
\label{chap:Hyperparameter Optimierung}
% Baysian Optimization. Hyperparameter Optimierung. Vlt. Geschwindigkeitsoptimierung des Algorithmus
Zur Optimierung der Hyperparameter wird der Baysian Algorithmus verwendet. Es
wird ein bereits implementierung in Python genutzt, um eine schnellere
Ergebnisse zu erzielen. \cite{fernando_bayesian_2022}

Als Blackboxfunktion wurde das Agent im Environment genommen. Dieser trainiert
$4000$ Episoden und der Durchschnitt des Rewards der letzten 12 Episoden wird
als Wert zur optimierung verwendet. Um eine bessere Vergleichbarkeit dieser
Werte sicherzustellen, werden immer dieselben $4000$ Referenzbilder gezeigt.
