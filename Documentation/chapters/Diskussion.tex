\chapter{Diskussion}
Die Diskussion analyisiert die Resultate der Methode (siehe \nameref{chap:m}), um daraus eine Antwort auf
die Fragestellung zu bilden. Zu diesem Zweck werden einige allgemeine
Feststellungen getroffen und die Unterfragen beantworten. Im zweiten Teil der
Diskussion folgt ein Fazit, ein Ausblick und eine Selbstreflexion. Dabei
verschiebt sich der Fokus von der Fragestellung weg und auf eine allgemeinere
Betrachtung der Arbeit.


\section{Fragestellung und Unterfragen}
- Allgemeine Festellungen

Anzumerken ist, dass der
Wert der Erkennbarkeit in allen Versionen für das EMNIST Datenset schlechter ist
als für die restlichen Datensets. Das hängt allerdings vermutlich mit der künstlichen Intelligenz zusammen, die die EMNIST Buchstaben erkennt, und nicht mit 


\subsection{Beantwortung der Unterfragen}
\label{sub:d_frage_unter}

Insgesamt sechs Unterfragen werden beantwortet {siehe Einleitung}. Diese
Unterfragen weiten die Fragestellung aus und tragen zu der schlussendlichen
Antwort auf die Fragestellung bei. Die Antworten beruhen auf den Resultaten aus
der Methode, also den Tabellen und Bildersammlungen, aber auch auf Erkenntnisen
aus der Methode selbst.

\subsubsection*{Wie kann die Architektur einer KI aussehen, die das Nachzeichnen erlernt?}
Unter der Annahme, dass die künstliche Intelligenz dieser Arbeit das
Nachzeichnen erlernt, {siehe Fragestellung}, kann die Architektur genau so
aussehen, wie sie in dieser Arbeit beschrieben ist. Das bedeutet durch
Reinforcement Learning und spezifischer durch Deep Q-Learning. Die genaue
Architektur ist in der Abbildung {abb} erkennbar.

\subsubsection*{Wie lässt sich die Leistung der KI in dieser Aufgabe beurteilen?}
Die Leistung der künstlichen Intelligenz lässt sich durch die definierten
Kriterien beurteilen. Diese wären Übereinstimmung, Erkennbarkeit und
Geschwindigkeit. Die Übereinstimmung ist ein objektiver und Absoluter Wert,
und somit das Aussagekräftigste Kriterium. die Kriterien der Erkennbarkeit
und Geschwindigkeit sind an subjektive Annahmen und an die Beurteilung durch
eine zweite künstlichen Intelligenz gebunden. Dadurch sinkt ihre
Aussagekraft. Allerdings ändern sich die subjektiven annahmen nicht und die
Kriterien sind in jedem Fall durch einen Zahlenwert repräsentiert. somit
eignen sich die Kriterien als vergleichbare Variablen zwischen Versionen der künstlichen Intelligenz

\subsubsection*{Wie lässt sich die Leistung der KI verbessern?}
Bezogen auf die definierten Kriterien erreicht die Grundversion Werte, die
durch die implementierten Variationen nicht oder nur marginal verbessert
werden. Die Grundversion erlebte allerdings in dessen Entwicklung
signifikante verbesserungen. Die grössten Verbesserungen stammen aus der
Optimierung der Hyperparamter durch den Baysian Algorithmus. Zum Beispiel
hat die Grösse des Replay Buffers einen erheblichen Effekt auf die Leistung.

\subsubsection*{Welche Einflüsse haben Physiksimulationen auf die Leistung der KI?}
Bezogen auf die definierten Kriterien verschlechtert sich die Leistung der KI.
Alle Versionen, die auf der Grundumgebung basieren, erzielen höhere Werte als
die gleichen Versionen basierend auf der phyiskalischen Umgebung. Die
physikalische Umgebung hat zum Ziel, die Bewegungen der KI realistischer zu
gestalten. In diesem Bereich kann der Einfluss nicht objektiv bestimmt werden,
da er nicht aus den Zahlenwerten der Kriterien hervorgeht. Aus Beobachtungen der
Bilder, welche in der physikalischen Umgebung gezeichnet sind, gehen ebenfalls
keine Erkenntnisse in diesem Bereich hervor. Die Bilder unterscheiden sich kaum
von denjenigen aus der Grundumgebung.

\subsubsection*{Wie ändert sich die Leistung der KI bei Strichbildern, die sich von den Trainingsdaten unterscheidenn}
In allen acht Versionen bleibt die Leistung der KI zwischen den drei Datensets
(siehe \autoref{tab:datasets}) vergleichbar. Die Tabellen {...} und {...} Zeigen
die Leistung der Grund-Basis Version und der Physik-Basis Version in den drei
definierten Kriterien, getestet auf die drei Datensets. Der Wert der
Übereinstimmung zwischen dem MNIST Datenset und dem EMNIST Datenset ist beinahe
identisch. Für beide Versionen ist der Wert der Übereinstimmung für das
QuickDraw Datenset niedriger. Insgesamt ist die KI in diesem Kriterium jedoch
kaum Beeinflusst durch die Wahl des Datensets. Die Analyse der anderen zwei
Kriterien führt zu einer ähnlichen Schlussfolgerung. Interessant ist, dass vor
allem die Grund-Basis Version eine viel höhere Geschwindigkeit im Zeichnen von
MNIST Zahlen hat, als im Zeichnen von EMNIST Buchstaben. Obwohl die Formen zu
grossem Teil ähnlich sind, scheint die KI durch das spezifische Training auf
MNIST Ziffern eine höhere Geschwindigkeit zu entwickeln. 



\subsubsection*{Inwiefern lässt sich das Zeichnen der KI mit menschlichem Zeichnen vergleichen?}
Die Antwort auf diese Frage leitet sich nicht aus den objektiven Resultaten
ab, sondern basiert auf subjektiven Beobachtungen. Die Bewegungen in der
Physik-Version der künstlichen Intelligenz basieren grundsätzlich auf den
selben Gesetzen wie die Bewegungen in der echten Welt. Allerdings sind die
Bewegungen stark vereinfacht. So kann auf den Agent in einer Zeichnung nur 64
eine Kraft (hier eine Beschleunigung) wirken, während die Krafzufuhr durch den
menschlichen Körper bedeutend vielseitiger ist. Ausserdem ist für die
künstliche Intelligenz der Druck des Stiftes nicht veränderbar. Zumindest
Konzeptuell aber nähert die künstliche Intelligenz menschliches Zeichnen,
Bezogen auf die physischen Einschränkungen, an. Einige menschliche
Gewohnheiten sind bei der künstlichen Intelligenz allerdings nicht
beobachtbar. Zum Beispiel beginnen Menschen in der Regel bei dem Zeichnen
einer Ziffer immer an der selben Stelle, oder zeichnen die Zahl von oben nach
unten. Die künstliche Intelligenz beginnt für jede Zeichnung an einem anderen
Ort. Aus Beobachtungen lässt sich schliessen, dass die künstliche Intelligenz
in der Regel an der Stelle mit dem geringsten Abstand zu der zufälligen
Startpostion mit dem Zeichnen beginnt.


\subsection{Beantwortung der Fragestellung}
Die Fragestellung lautet: in wiefern kann eine künstliche Intelligenz lernen,
Strichbilder auf eine physische Weise nachzuzeichnen, sodass diese durch ein
automatisches System erkannt werden? {siehe Einleitung} Diese Frage hat mehrere
Aspekte, die teilweise bereits durch die Unterfragen erfasst werden. Für die
schlussendliche Antwort folgt eine genauere Ausführung dieser Aspekte.

Die künstliche Intelligenz zeichnet durch Physiksimulationen und durch allgemeine
Einschränkungen der Bewegungsfreiheit auf eine annähernd physische Weise. Das
Zeichnen ist nur annähernd physisch, da alle Bewegungen simuliert und in
keiner phyischen Umgebung umgesetzt sind. Ausserdem sind die Simulationen
nicht vollkommen realitätsgetreu {siehe Diskussion -> Einflüsse Phyisk }

Die künstliche intelligenz erlernt das Nachzeichnen bezogen auf die Kriterien,
in denen es definiert ist, erfolgreich {Siehe Meth. Kriterien}. Dafür sprechen
die Werte der besten Versionen für das Nachzeichnen von Ziffern, die teilweise
an den Höchstwert grenzen {siehe Tabelle beste Versionen}. Die hohen Werte im
Kriterium der Erkennbarkeit bestätigen ausserdem, dass die Zeichnungen der
künstlichen Intelligenz von einem automatischen System erkannt werden, zumindest
in der Mehrheit der Fälle und für das Zeichnen von Ziffern.


Laut der Fragestellung soll die KI das Nachzeichnen von Strichbildern erlernen.
Damit ist implizit das Nachzeichnen von allen möglichen Arten von Strichbildern
gemeint. Die Leistung der KI kann nicht auf alle möglichen Strichbilder
überprüft werden, aber der Test mit drei verschiedenen Datensets ergibt
vielversprechende Resultate. Die KI erlernt erfolgreich das Nachzeichnen von
Ziffern, Kleinbuchstaben und zehn zufälligen Motiven aus dem QuickDraw Datenset.
Durch die Vielfalt im QuickDraw Datenset kann die Annahme getroffen werden, dass
die KI zumindest einen grossen Teil an Strichbildern nachzeichnen kann. 

Die zusammenfassende Antwort auf die Frage lautet somit: Eine künstliche
Intelligenz kann das Nachzeichnen von Strichbildern auf annähernd physiche Weise
in dem Sinne lernen, dass die fertige Zeichnung von einem automatischen System
grösstenteils erkannt wird, Die Übereinstimmung zwischen der Vorlage und der
Zeichnung gross ist und die Zeichnung nicht viel Zeit in Anspruch nimmt.

Diese Antwort bezieht sich auf die genau Frage, wie sie in der Einleitung steht.
Der nächste Abschnitt beurteilt die Frage durch die Erkenntnisse aus dieser
Arbeit und geht auf mögliche Erweiterungen ein.



\section{Fazit und Ausblick}
Die Resultate erlauben eine positive Antwort auf die Fragestellung. Diese
Antwort setzt allerdings einige Annahmen vorraus, die weiter diskutiert werden
können. Die grösste Annahme bezieht sich auf die Definition von Nachzeichnen.
Diese Arbeit definiert Nachzeichnen durch drei Kriterien und durch physische
Rahmenbedingungen. Die Kriterien sind für eine künstliche Intelligenz sinnvoll
gewählt {siehe wie lässt sich Leistung beurteilen}. Allerdings wären auch
andere, teilweise subjektive Kriterien möglich, die das Nachzeichnen definieren.
Nachzeichnen ist eine menschliche Tätigkeit. Dadurch spielt die menschliche
Wahrnehmung eine grosse Rolle, vor allem bei den genauen Bewegungsabläufen und
den Pfaden, auf denen gezeichnet wird. Der künstlichen Intelligenz fehlt eine
derartige Wahrnehmung, wodurch sich dessen Bewegungen im Vergleich zu einem
Menschen stark unterscheiden. {siehe vergleich menschliches Zeichnen}.
Kriterien, die sich auf diesen menschlichen Aspekt des Nachzeichnens beziehen
wären eine mögliche Weiterführung dieser Arbeit. Dabei stellt sich allerdings
die Frage, wie die künstliche Intelligenz die Leistung eines subjektiven
Kriteriums maximieren kann.


Die physichen Rahmenbedingungen unterscheiden sich von denjenigen, die ein
Mensch erfährt. Das kommt daher, dass die phyischen Rahmenbedingungen im Falle
der künstlichen Intelligenz lediglich simuliert sind. Das verunmöglicht eine
umfassende Antwort auf die Frage, ob die künstliche Intelligenz auf eine
physische Weise zeichnet. Dieses Problem könnte mit einem Roboter gelöst werden,
der die künstliche Intelligenz in eine reale, phyische Umgebung überführt. Der
Roboter könnte somit verschiedenste Strichbilder auf einem echten Stück Papier,
und somit zwangsläufig auf physische Weise nachzeichnen. Aktuell sind die
Bewegungen der künstlichen Intelligenz in gewissen Belangen eingeschränkt. So
ist beispielsweise die Druckstärke nicht variiierbar. Ausserdem zeichnet die
künstliche Intelligenz vorwiegend kleine Strichbilder. Experimente mit grösseren
Konstrukten, wie ganze Wörter, wären eine mögliche Erweiterung. 

Alles in allem sind eine Vielzahl an denkbaren Fragen und Ideen möglich, die auf
ReSketch, der künstlichen Intelligenz hinter dieser Arbeit, basieren.




\section{Selbstreflexion}
Die Selbstreflexion gibt genauere Einblicke in die Vorangehensweise hinter
dieser Arbeit. Die Arbeit ist grundsätzlich eine Zusammenfassung der wichtigsten
Ereignisse. Viele Aspekte, wie auch die Arbeitsweise bleiben verschwiegen. Die
selbstreflexion geht näher auf drei wichitige Aspekte ein, die in der
zusammengefassten Arbeit nicht genug betont sind.

\subsection{Optimierung der künstlichen Intelligenz}
Insgesamt sind acht Versionen der KI präsentiert. Im Verlaufe des Projektes gab
es viele weitere Versuche, die Leistung der künstlichen Intelligenz zu
verbessern. Diese Versuche führten Allerdings häufig dazu, dass die künsltiche
Intelligenz garnicht mehr lernt. In der Arbeit sind deswegen nur diejenigen
Versuche vermerkt, die tatsächlich funktionieren. Ein Beispiel für einen
gescheiterten Versuch wäre die Anpassung der Reward-Function, sodass diese nur
bei der korrekten Erkennung einer Zahl einen Reward auslöst. Keine
ausprobierte Anpassung in der Architektur der künstlichen Intelligenz macht
diesen Versuch funktionstüchtig. Das Problem von derartigen Versuchen liegt
darin, dass die Ursache hinter ihrem Scheitern oder ihrem Erfolg häufig nicht
erkennbar ist. Das macht die Optimierung der künstlichen Intelligenz allgemein
schwierig. 

Interessanterweise übertrifft keine präsentierte Variation der künstlichen
Intelligenz das Grundprogramm. Die Variationen können somit die Leistung der
künstlichen Intelligenz nicht optimieren. Die simpelste Version stellt sich als
die leistungsfähigste heraus. Das hat unter anderem damit zu tun, dass die
Leistung des Grundprogrammes bereits nahe am erreichbaren Maximum ist.

Die Strategie hinter der Optimierung besteht in den meisten Fällen aus
wiederholtem Ausprobieren mit Anpassungen zwischen jedem Versuch. Hilfsmittel,
wie der Baysian Algorithmus {siehe Hyperparameteroptimierung}, vereinfachen
diese Aufgabe stark. Tatsächlich ermöglichte der Baysian Algorithmus eine
Verbesserung der künstlichen Intelligenz von ungefähr $20-30\%$ Übereinstimmung
auf die maximalen Werte, die sie schlussendlich erreicht. Diese Strategie der
Optimierung ist für einen Computer sehr ressourcenintensiv. In den längsten
Optimierungsarbeiten liefen die beiden Computer, auf denen die Arbeit
verrichtet wurde, jeweils länger als $24$ Stunden.



\subsection{Analyse der künstlichen Intelligenz}
Eine Analyse der künstlichen Intelligenz ist notwendig, um die Fragestellung und
die Unterfragen zu beantworten. Aber auch während der Entwicklung der
künsltichen Intelligenz ist eine stetige Analyse nötig, um dise zu verstehen und
zu verbessern.

Die Analyse besteht hauptsächlich darin, die Leistung der künstlichen
Intelligenz zu beurteilen. Das geschieht mittels den Kriterien, die für diesen
Zweck definiert sind {siehe Kriterien}. Die Kriterien sind dabei so definiert,
dass sie für jede mögliche Variation gleich gemessen werden. Der durchnittliche
akkumulierte Reward wird Beispielsweise bewusst nicht als Kriterium verwendet.
Der akkumulierte Reward {siehe Theorie reinf} ist abhängig von der
Reward-Function und unterscheidet sich somit zwischen Variationen. Dieser ist
somit nicht für einen Vergleich zwischen Variationen geeignet.

Einzelne Variationen vergleichbar zu halten ist ein allgemeines Ziel der
Analyse. Deswegen basieren alle Variationen auf der gleichen Architektur der
künstlichen Intelligenz. Die Funktionalität des Grundprogrammes {siehe
Grundprogramm} ist gründlich getestet. Zum Beispiel sind die Testdaten darauf
überprüft, dass diese keine Datenpunkte aus den Trainingsdaten beinhalten. Diese
Tests eliminieren mögliche Fehlerquellen salls die künstliche Intelligenz
unerwartetes Verhalten aufzeigt.

Eine weitere Form von Analyse stammt aus der Sammlung von Daten über das
Lernverhalten der künstlichen Intelligenz. So wird aus jedem Training ein Graph
erstellt, der die durchschnittliche Leistung der künstlichen Intelligenz in
jeder Episode erfasst {Siehe Abbildungen Graphen}. Die Leistung ist dabei durch
den akkumulierten Reward pro Episode repräsentiert. Wie erwähnt können Versionen
der künsltihcen Intelligenz nicht anhand ihres akkumulierten Rewards verglichen
werden. Der akkumulierte Reward zeigt allerdings für einzelne Versionen am
präzistesten, in wiefern diese ihren Reward maximieren können.




\subsection{Verwendung von Git und GitHub}
Ohne die Verwendung von Git und Github ist ein Projekt von diesem Ausmass nur
schwer umsetzbar. Die Programme ermöglichen einfache Zusammenarbeit am
Programmcode und an der Dokumentation. GitHub dient dabei zusätzlich als
Hilfsmiitel zur Organisation durch die integrierte Funktion der Project Boards.
Diese Funktion hätte allerdings zu grösserem Ausmass Verwendung finden können.

Die Funktion der Branches und Commits von Git werden durch die Arbeit hindurch
konsequent verwendet. So liegt der Ursprung jeder Variation der künstlichen
Intelligenz auf einem eigenen Branch. Auch die verschiedenen Kapitel der
Dokumentation sind jeweils in einem eigenen Branch verfasst. Die Verzweigung des
Projektes in Branches ermöglicht eine bessere Ordnung und Strukturierung des
Projektes. Wenn die Arbeit an einem Branch erledigt ist, wird dieser wieder mit
dem Main Branch zusammengeführt. Für die wichtigsten Branches wird dabei das
Prinzip der Pull Request angewendet. Die Pull Request muss für jeden Branch von
beiden Autoren akzeptiert werden.

Ein weiterer Vorteil von Git und Github ist die Zugänglichkeit des Projektes.
Das gesamte Projekt ist unter folgendem Link einsehbar:
https://github.com/LarsZauberer/Nachzeichner-KI. Im Projektordner sind
vortrainierte Variationen der künstlichen Intelligenz enthalten, mit denen die
diese getestet werden können. Das Projekt auf GitHub erfährt möglicherweise
Erweiterungen, die in dieser Arbeit nicht mehr erfasst sind.






