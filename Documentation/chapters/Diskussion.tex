\chapter{Diskussion}

\section{Interpretation der Resultate}
Warum funktionieren einige Versionen besser als andere

\subsection*{Ergebnisse des Trainings mit Schrifterkennung}
- Wurde KI besser? In welchen Kriterien?
- Warum nicht?
- läuft es besser auf Basisversion oder Physik

- Hilfreich für Speed? (Vergleich Speed Training, Basisversion Speed Test)

\subsection*{Ergebnisse der Physikalischen Umgebung}
- Vergleich zu Basisversion: Schlechter oder besser
- Vergleich durchschnittsspeed. Wie kann Verglichen werden (andere Skala)
- Probleme der Physikalischen Umgebung: Bump in Wand

\section{Vergleich zu menschlichem Zeichnen}
- Einziger Anhaltspunkt in Gifs, Videos

- Stiftbewegungen realistischer durch Physik?
Physikalische Entfernung von Realität? -> 2D Kein Druck auf Stift -> mögliche Erweiterung

- Geschwindigkeit der Bewegung
    (Möglich: Zwar schnell aber nicht realistisch, Springt hin und her, Physik prioritiert Geschwindigkeit in gerader Linie über Pixelgenauigkeit)

- Limitierungen
    Unfähig, andere Symbole zu zeichnen. Nur Zahlenwerte
    Sehr unterschiedliche Ergebnisse


\section{Selbstreflexion}
In der Selbstreflexion sollen verschiedene Aspekte in der Entwicklung dieses
Projektes betrachtet und reflektiert werden. Es wird ein besonderer Wert auf 3
Nachfolgende Teilgebeite dieser Arbeit gelegt, welche zu beginn des Projektes in
der Projektvereinbarung mit der Neuen Kantonsschule Aarau festgelegt wurden.

Allgemein betrachtet verlief das Projekt relativ reibungslos und ohne grosse
Schwierigkeiten. Teilweise gab es Verzögerungen, weil nicht so gute Ansätze oder
Technologien eingesetzt wurden. Dennoch wurden diese schnell entdeckt und
verbessert. Ausserdem bleiben diese Art von Problemen bei einer solchen Arbeit
nicht aus.

% BUG: Präteritum?

\subsection*{Verwendung von Git und GitHub}
In diesem Teil dieser Arbeit war es das Ziel möglichst gut Git und GitHub (siehe
theorie git+github)  % TODO: Ref setzen
als eine Möglichkeit zur Organisation und zur Kommunikation des Projektes zu
nutzen. Dabei soll Git und GitHub möglichst konsequent und umfangreich genutzt
werden.

Git und GitHub waren eine grosse Bereicherung für diese Arbeit. Die
Nutzung dieser Programme ermöglichte einfache Zusammenarbeit am Code und auch an
der Dokumentation. Nicht nur die Zusammenarbeit, sondern auch die Organisation
konnte dadurch vereinfacht werden. So wurden die GitHub Project Boards
eingesetzt um eine Übersicht über die noch zu erledigende Arbeit zu erhalten.

Git wurde sehr gut eingesetzt. Branches und Commits wurden konsequent genutzt.
Durch die Branches und Commits konnten einfach Vergleiche zwischen den Versionen
gemacht werden, ohne ein Datei Chaos zu haben.

Diese Branches wurden dann mit GitHub durch Pull Requests mit weiter tiefere
Branches vermischt. Dadurch hat immer eine zweite Person den Code oder den neuen
Text der Dokumentation angeschaut. So konnte eine gute Qualität garantiert
werden.

% Die Empfohlene Branch Struktur (siehe theorie git) wurde zwar grösstenteils  % TODO: Ref setzen
% eingehalten, war aber in dieser Art Arbeit eher weniger brauchbar. Das liegt
% daran, dass die Struktur vor allem gut für Produkte und nicht für
% wissenschaftliche Arbeiten ist. In dieser Arbeit gibt es mehrere Versionen, die
% Gleichberechtigt sind und nicht miteinander kompatibel sind. Somit können sie
% eigentlich nicht in eine einzelne Version zusammengemischt werden.
% TODO: Die Branch Struktur wird glaube ich nicht im Theorieteil erwähnt.

Zu verbessern wäre noch eine verstärktere Nutzung der Issues und der Project
Boards von GitHub. Diese hätten vielleicht noch etwas mehr genutzt werden
können, um eine noch organisierte Struktur zu erhalten.

% TODO: Sollen hier auch noch ein paar GitHub Features aufgezählt werden, welche wir nicht genutzt haben.


\subsection*{Optimierung der KI}
In diesem Ziel geht es darum, wie gross die Bemühungen/Erfolge die KI zu
optimieren waren.

In dieser Arbeit wurde sehr viel Wert auf die Verbesserung der Leistung der KI
gelegt. Es wurden verschiedene Messwerte (siehe resultate Messwerte) erarbeitet,  % TODO: Ref setzen - Ref zu Resultaten oder in die Methode?
wie die Leistung der KI Beurteilt werden kann.

Auch durch die Hyperparameteroptimierung (siehe methode Hyperparameteroptimierung) % TODO: Ref setzen
konnte die Leistung der KI stark verbessert werden.

Am Ende hat die Arbeit die Leistung der Vorgängerarbeit (DoodleSDQ) (siehe theorie  % TODO: Ref setzen
DoodleSDQ) übertroffen. Allerdings ist hier wichtig zu erwähnen, dass im
DoodleSDQ die Aufgabe schwieriger war und nicht ganz die selben Ziele verfolgt
wurden. Beispielsweise sind in dieser Arbeit die Zeichnungen kleiner und weniger
komplex. Es wurde sich in dieser Arbeit nur auf Zahlen fokusiert.


\subsection*{Analyse der KI}
Das Ziel beim Teilgebiet der Analyse der KI ist es möglichst genau
nachvollziehen zu können, wieso die KI ein bestimmtes Verhalten hat und diese
genau in der Dokumentation zu beschreiben. Zusätzlich wie gut aus diesen
Informationen dann Schlussfolgerungen gezogen wurden.

Diese Arbeit bemühte sich sehr um eine möglichst genaue Beschreibung und Analyse
der KI. Schlussfolgerungen und Beschreibungen des Verhaltens können in Folgenden Kapiteln erfasst werden.

\begin{itemize}
    \item 
\end{itemize}
% TODO: Refs setzen
