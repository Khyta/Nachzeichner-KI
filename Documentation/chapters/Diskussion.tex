\chapter{Diskussion}
% - Einleitung

\section{Fragestellung und Unterfragen}
% - Allgemeine Festellungen
% - Transition: Beantwortung der Unterfragen aus der Einleitung -> Das ermöglicht Beantwortung der eigentlichen Fragestellung



\subsection{Beantwortung der Unterfragen}
% - 6 Unterfragen
% - Teilweise Beantwortbar durch: Statistiken, Bilder, Erkenntnisse aus der Methode

\subsubsection*{Wie kann die Architektur einer KI aussehen, die das Nachzeichnen erlernt?}
% - Unter der Annahme, dass die künstliche Intelligenz dieser Arbeit das
%   Nachzeichnen erlernt, <siehe Fragestellung>, kann die Architektur genau so
%   aussehen, wie sie in dieser Arbeit beschrieben ist. Das bedeutet durch
%   Reinforcement Learning und spezifischer durch Deep Q-Learning. Die genaue
%   Architektur ist in der Abbildung <abb> erkennbar.

\subsubsection*{Wie lässt sich die Leistung der KI in dieser Aufgabe beurteilen?}
% - Die Leistung der künstlichen Intelligenz lässt sich durch die definierten
%   Kriterien beurteilen. Diese wären Übereinstimmung, Erkennbarkeit und
%   Geschwindigkeit. Die Übereinstimmung ist ein objektiver und Absoluter Wert,
%   und somit das Aussagekräftigste Kriterium. die Kriterien der Erkennbarkeit
%   und Geschwindigkeit sind an subjektive Annahmen und an die Beurteilung durch
%   eine zweite künstlichen Intelligenz gebunden. Dadurch sinkt ihre
%   Aussagekraft. Allerdings ändern sich die subjektiven annahmen nicht und die
%   Kriterien sind in jedem Fall durch einen Zahlenwert repräsentiert. somit
%   eignen sich die Kriterien als vergleichbare Variablen zwischen Versionen der künstlichen Intelligenz


\subsubsection*{Wie lässt sich die Leistung der KI verbessern?}
% - Bezogen auf die definierten Kriterien erreicht die Grundversion Werte, die
%   durch die implementierten Variationen nicht oder nur marginal verbessert
%   werden. Die Grundversion erlebte allerdings in dessen Entwicklung
%   signifikante verbesserungen. Die grössten Verbesserungen stammen aus der
%   Optimierung der Hyperparamter durch den Baysian Algorithmus. Zum Beispiel
%   hat die Grösse des Replay Buffers einen erheblichen Effekt auf die Leistung.



\subsubsection*{Welche Einflüsse haben Physiksimulationen auf die Leistung der KI?}
% -Bezogen auf die definierten Kriterien verschlechtert sich die Leistung. Die
% Implementierung von Physiksimulationen definiert die Rahemenbedingungen des
% Nachzeichnens neu und stellt den Versuch dar, die Bewegungen realistischer zu
% gestalten. In diesem Bereich kann der Einfluss nicht objektiv bestimmt werden,
% da er nicht aus den Zahlenwerten der Kriterien hervorgeht. Der Vergleich
% zwischen nachgezeichneten Bildern aus der Grundversion und der physikalischen Version führt zu
% folgenden Aussagen:


\subsubsection*{Wie ändert sich die Leistung der KI bei Strichbildern, die sich von den Trainingsdaten unterscheidenn}
% - Wieso sind einige Versionen besser als andere

\subsubsection*{Inwiefern lässt sich das Zeichnen der KI mit menschlichem Zeichnen vergleichen?}
% Die Antwort auf diese Frage leitet sich nicht aus den objektiven Resultaten
% ab, sondern basiert auf subjektiven Beobachtungen. Die Bewegungen in der
% Physik-Version der künstlichen Intelligenz basieren grundsätzlich auf den
% selben Gesetzen wie die Bewegungen in der echten Welt. Allerdings sind die
% Bewegungen stark vereinfacht. So kann auf den Agent in einer Zeichnung nur 64
% eine Kraft (hier eine Beschleunigung) wirken, während die Krafzufuhr durch den
% menschlichen Körper bedeutend vielseitiger ist. Ausserdem ist für die
% künstliche Intelligenz der Druck des Stiftes nicht veränderbar. Zumindest
% Konzeptuell aber nähert die künstliche Intelligenz menschliches Zeichnen,
% Bezogen auf die physischen Einschränkungen, an. Einige menschliche
% Gewohnheiten sind bei der künstlichen Intelligenz allerdings nicht
% beobachtbar. Zum Beispiel beginnen Menschen in der Regel bei dem Zeichnen
% einer Ziffer immer an der selben Stelle, oder zeichnen die Zahl von oben nach
% unten. Die künstliche Intelligenz beginnt für jede Zeichnung an einem anderen
% Ort. Aus Beobachtungen lässt sich schliessen, dass die künstliche Intelligenz
% in der Regel an der Stelle mit dem geringsten Abstand zu der zufälligen
% Startpostion mit dem Zeichnen beginnt.


\subsection{Beantwortung der Fragestellung}
Die Fragestellung, wie in der Einleitung lautet: in wiefern kann eine künstliche
Intelligenz lernen, Strichbilder auf eine physische Weise nachzuzeichnen, sodass
diese durch ein automatisches System erkannt werden? Diese Frage hat mehrere
Aspekte, die teilweise bereits durch die Unterfragen erfasst werden. Für die
schlussendliche Antwort folgt eine genauere Ausführung dieser Aspekte.

Die künstliche Intelligenz zeichnet durch Physiksimulationen und allgemeinen
Einschränkungen der Bewegungsfreiheit auf eine annähernd physische Weise. Das
Zeichnen ist nur annähernd physisch, da alle Bewegungen simuliert und in
keiner phyischen Umgebung umgesetzt sind. Ausserdem sind die Simulationen
nicht vollkommen realitätsgetreu {siehe Diskussion -> Einflüsse Phyisk }

Die künstliche intelligenz erlernt das Nachzeichnen bezogen auf die Kriterien,
in denen es definiert ist, erfolgreich {Siehe Meth. Kriterien}. Dafür sprechen
die Werte der besten Versionen für das Nachzeichnen von Ziffern, die teilweise
an den Höchstwert grenzen {siehe Tabelle beste Versionen}. Die hohen Werte im
Kriterium der Erkennbarkeit bestätigen ausserdem, dass die Zeichnungen der
künstlichen Intelligenz von einem automatischen System erkannt werden, zumindest
in der Mehrheit der Fälle und für das Zeichnen von Ziffern.

% Lernt die künstliche Intelligenz das Nachzeichnen von Strichbildern Allgemein?
% -> Vergleich MNIST, EMNIST, QuickDraw. Durch grosse Varianz in Quickdraw kann
% die Annahme getroffen werden, die künstliche Intelligenz das Nachzeichnen von
% Strichbildern allgemein lernt.

Die zusammenfassende Antwort auf die Frage lautet somit: Eine künstliche
Intelligenz kann das Nachzeichnen von Strichbildern auf annähernd physiche Weise
in dem Sinne lernen, dass die fertige Zeichnung von einem automatischen System
grösstenteils erkannt wird, Die Übereinstimmung zwischen der Vorlage und der
Zeichnung gross ist und die Zeichnung nicht viel Zeit in Anspruch nimmt.

Diese Antwort bezieht sich auf die genau Frage, wie sie in der Einleitung steht.
Der nächste Abschnitt beurteilt die Frage durch die Erkenntnisse aus dieser
Arbeit und geht auf mögliche Erweiterungen ein.



\section{Fazit und Ausblick}
Die Resultate erlauben eine positive Antwort auf die Fragestellung. Diese
Antwort setzt allerdings einige Annahmen vorraus, die weiter diskutiert werden
können. Die grösste Annahme bezieht sich auf die Definition von Nachzeichnen.
Diese Arbeit definiert Nachzeichnen durch drei Kriterien und durch physische
Rahmenbedingungen. Die Kriterien sind für eine künstliche Intelligenz sinnvoll
gewählt {siehe wie lässt sich Leistung beurteilen}. Allerdings wären auch
andere, teilweise subjektive Kriterien möglich, die das Nachzeichnen definieren.
Nachzeichnen ist eine menschliche Tätigkeit. Dadurch spielt die menschliche
Wahrnehmung eine grosse Rolle, vor allem bei den genauen Bewegungsabläufen und
den Pfaden, auf denen gezeichnet wird. Der künstlichen Intelligenz fehlt eine
derartige Wahrnehmung, wodurch sich dessen Bewegungen im Vergleich zu einem
Menschen stark unterscheiden.

Die physichen Rahmenbedingungen unterscheiden sich ebenfalls von denjenigen, die
ein Mensch erfährt. Das kommt daher, dass die phyischen Rahmenbedingungen im
Falle der künstlichen Intelligenz lediglich simuliert sind. Durch einen Roboter
könnte die künstliche Intelligenz in eine reale, phyische Umgebung überführt
werden. Der Roboter könnte somit verschiedenste Strichbilder auf einem echten
Stück Papier nachzeichnen. 
Aktuell sind die Bewegungen der künstlichen
Intelligenz in gewissen Belangen eingeschränkt. So kann sie zum Beispiel die
Druckstärke nicht variieren. Ausserdem zeichnet die künstliche Intelligenz
vorwiegend kleine Strichbilder. Experimente mit grösseren Konstrukten, wie ganze
Wörter, wären eine mögliche Variation





\section{Selbstreflexion}
In der Selbstreflexion sollen verschiedene Aspekte in der Entwicklung dieses
Projektes betrachtet und reflektiert werden. Es wird ein besonderer Wert auf 3
Nachfolgende Teilgebeite dieser Arbeit gelegt, welche zu beginn des Projektes in
der Projektvereinbarung mit der Neuen Kantonsschule Aarau festgelegt wurden.

Allgemein betrachtet verlief das Projekt relativ reibungslos und ohne grosse
Schwierigkeiten. Teilweise gab es Verzögerungen, weil nicht so gute Ansätze oder
Technologien eingesetzt wurden. Dennoch wurden diese schnell entdeckt und
verbessert. Ausserdem bleiben diese Art von Problemen bei einer solchen Arbeit
nicht aus.

% BUG: Präteritum?

\subsection*{Verwendung von Git und GitHub}
In diesem Teil dieser Arbeit war es das Ziel möglichst gut Git und GitHub (siehe
theorie git+github)  % TODO: Ref setzen
als eine Möglichkeit zur Organisation und zur Kommunikation des Projektes zu
nutzen. Dabei soll Git und GitHub möglichst konsequent und umfangreich genutzt
werden.

Git und GitHub waren eine grosse Bereicherung für diese Arbeit. Die
Nutzung dieser Programme ermöglichte einfache Zusammenarbeit am Code und auch an
der Dokumentation. Nicht nur die Zusammenarbeit, sondern auch die Organisation
konnte dadurch vereinfacht werden. So wurden die GitHub Project Boards
eingesetzt um eine Übersicht über die noch zu erledigende Arbeit zu erhalten.

Git wurde sehr gut eingesetzt. Branches und Commits wurden konsequent genutzt.
Durch die Branches und Commits konnten einfach Vergleiche zwischen den Versionen
gemacht werden, ohne ein Datei Chaos zu haben.

Diese Branches wurden dann mit GitHub durch Pull Requests mit weiter tiefere
Branches vermischt. Dadurch hat immer eine zweite Person den Code oder den neuen
Text der Dokumentation angeschaut. So konnte eine gute Qualität garantiert
werden.

% Die Empfohlene Branch Struktur (siehe theorie git) wurde zwar grösstenteils  % TODO: Ref setzen
% eingehalten, war aber in dieser Art Arbeit eher weniger brauchbar. Das liegt
% daran, dass die Struktur vor allem gut für Produkte und nicht für
% wissenschaftliche Arbeiten ist. In dieser Arbeit gibt es mehrere Versionen, die
% Gleichberechtigt sind und nicht miteinander kompatibel sind. Somit können sie
% eigentlich nicht in eine einzelne Version zusammengemischt werden.
% TODO: Die Branch Struktur wird glaube ich nicht im Theorieteil erwähnt.

Zu verbessern wäre noch eine verstärktere Nutzung der Issues und der Project
Boards von GitHub. Diese hätten vielleicht noch etwas mehr genutzt werden
können, um eine noch organisierte Struktur zu erhalten.

% TODO: Sollen hier auch noch ein paar GitHub Features aufgezählt werden, welche wir nicht genutzt haben.


\subsection*{Optimierung der KI}
In diesem Ziel geht es darum, wie gross die Bemühungen/Erfolge die KI zu
optimieren waren.

In dieser Arbeit wurde sehr viel Wert auf die Verbesserung der Leistung der KI
gelegt. Es wurden verschiedene Messwerte (siehe resultate Messwerte) erarbeitet,  % TODO: Ref setzen - Ref zu Resultaten oder in die Methode?
wie die Leistung der KI Beurteilt werden kann.

Auch durch die Hyperparameteroptimierung (siehe methode Hyperparameteroptimierung) % TODO: Ref setzen
konnte die Leistung der KI stark verbessert werden.

Am Ende hat die Arbeit die Leistung der Vorgängerarbeit (DoodleSDQ) (siehe theorie  % TODO: Ref setzen
DoodleSDQ) übertroffen. Allerdings ist hier wichtig zu erwähnen, dass im
DoodleSDQ die Aufgabe schwieriger war und nicht ganz die selben Ziele verfolgt
wurden. Beispielsweise sind in dieser Arbeit die Zeichnungen kleiner und weniger
komplex. Es wurde sich in dieser Arbeit nur auf Zahlen fokusiert.


\subsection*{Analyse der KI}
Das Ziel beim Teilgebiet der Analyse der KI ist es möglichst genau
nachvollziehen zu können, wieso die KI ein bestimmtes Verhalten hat und diese
genau in der Dokumentation zu beschreiben. Zusätzlich wie gut aus diesen
Informationen dann Schlussfolgerungen gezogen wurden.

Diese Arbeit bemühte sich sehr um eine möglichst genaue Beschreibung und Analyse
der KI. Schlussfolgerungen und Beschreibungen des Verhaltens können in Folgenden Kapiteln erfasst werden.

\begin{itemize}
    \item 
\end{itemize}
% TODO: Refs setzen
