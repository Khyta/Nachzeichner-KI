\chapter{Diskussion}

\section{Interpretation der Resultate}
Warum funktionieren einige Versionen besser als andere

\subsection*{Ergebnisse des Trainings mit Schrifterkennung}
- Wurde KI besser? In welchen Kriterien?
- Warum nicht?
- läuft es besser auf Basisversion oder Physik

- Hilfreich für Speed? (Vergleich Speed Training, Basisversion Speed Test)

\subsection*{Ergebnisse der Physikalischen Umgebung}
- Vergleich zu Basisversion: Schlechter oder besser
- Vergleich durchschnittsspeed. Wie kann Verglichen werden (andere Skala)
- Probleme der Physikalischen Umgebung: Bump in Wand

\section{Vergleich zu menschlichem Zeichnen}
- Einziger Anhaltspunkt in Gifs, Videos

- Stiftbewegungen realistischer durch Physik?
Physikalische Entfernung von Realität? -> 2D Kein Druck auf Stift -> mögliche Erweiterung

- Geschwindigkeit der Bewegung
    (Möglich: Zwar schnell aber nicht realistisch, Springt hin und her, Physik prioritiert Geschwindigkeit in gerader Linie über Pixelgenauigkeit)

- Limitierungen
    Unfähig, andere Symbole zu zeichnen. Nur Zahlenwerte
    Sehr unterschiedliche Ergebnisse


\section{Selbstreflexion}
In der Selbstreflexion sollen verschiedene Aspekte in der Entwicklung dieses
Projektes betrachtet und reflektiert werden. Es wird ein besonderer Wert auf 3
Nachfolgende Teilgebeite dieser Arbeit gelegt, welche zu beginn des Projektes in
der Projektvereinbarung mit der Neuen Kantonsschule Aarau festgelegt wurden.

Allgemein betrachtet verlief das Projekt relativ reibungslos und ohne grosse
Schwierigkeiten. Teilweise gab es Verzögerungen, weil nicht so gute Ansätze oder
Technologien eingesetzt wurden. Dennoch wurden diese schnell entdeckt und
verbessert. Ausserdem bleiben diese Art von Problemen bei einer solchen Arbeit
nicht aus.

% BUG: Präteritum?

\subsection*{Verwendung von Git und GitHub}
Zur Unterstützung dieser Arbeit wurden Programme, wie Git und GitHub (siehe
theorie git+github)  % TODO: Ref setzen
verwendet. Git und GitHub waren eine grosse Bereicherung für diese Arbeit. Die
Nutzung dieser Programme ermöglichte einfache Zusammenarbeit am Code und auch an
der Dokumentation. Nicht nur die Zusammenarbeit, sondern auch die Organisation
konnte dadurch vereinfacht werden. So wurden die GitHub Project Boards
eingesetzt um eine Übersicht über die noch zu erledigende Arbeit zu erhalten.

Git wurde sehr gut eingesetzt. Es gab sehr viele verschiedene Branches, als auch
Commits. Durch die Branches und Commits konnten einfach Vergleiche zwischen den
Versionen gemacht werden, ohne ein Datei Chaos zu haben.

Diese Branches wurden dann mit GitHub durch Pull Requests mit weiter tiefere
Branches vermischt. Dadurch hat immer eine zweite Person den Code oder den neuen
Text der Dokumentation angeschaut. So konnte eine gute Qualität garantiert
werden.

Die Empfohlene Branch Struktur (siehe theorie git) wurde zwar grösstenteils  % TODO: Ref setzen
eingehalten, war aber in dieser Art Arbeit eher weniger brauchbar. Das liegt
daran, dass die Struktur vor allem gut für Produkte und nicht für
wissenschaftliche Arbeiten ist. In dieser Arbeit gibt es mehrere Versionen, die
Gleichberechtigt sind und nicht miteinander kompatibel sind. Somit können sie
eigentlich nicht in eine einzelne Version zusammengemischt werden.

Zu verbessern wäre noch eine verstärktere Nutzung der Issues und der Project
Boards von GitHub. Diese hätten vielleicht noch etwas mehr genutzt werden
können, um eine noch organisierte Struktur zu erhalten.


\subsection*{Optimierung der KI}


\subsection*{Analyse der KI}
