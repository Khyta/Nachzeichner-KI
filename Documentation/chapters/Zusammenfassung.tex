\chapter{Zusammenfassung}\label{zusammenfassung}
Diese Untersuchung beantwortet die Frage, inwiefern eine künstliche Intelligenz
Strichbilder auf eine physische Weise nachzeichnen kann, sodass diese durch ein
automatisches System erkannt werden. Mit Strichbildern sind in diesem
Fall Ziffern aus dem MNIST Datenset, Buchstaben aus dem EMNIST Datenset und
weitere Motive aus dem QuickDraw Datenset gemeint. 

Zur Beantwortung der Fragestellung wird der Begriff des Nachzeichnes definiert.
Zu der Defintion gehören die Rahmenbedingungen, nach denen eine Tätigkeit als
Nachzeichnen gilt, und die Kriterien, die die Leistung im Nachzeichnen
beurteilen. Zu den Rahmenbedingungen gehören unter anderem die phyischen
Einschränkungen und die ausführbaren Aktionen der KI. Um die Leistung der KI im
Nachzeichnen zu bewerten, sind drei Kriterien definiert: Die Übereinstimmung der
Pixel, die Erkennbarkeit der Zeichnung und die Geschwindigkeit des Zeichnens.
Die Erkennbarkeit der Zeichnung wird durch eine zweite künstliche Intelligenz
ermittelt.

Das Ziel ist es, eine künstliche Intelligenz zu entwickeln, welche die gesetzten
Rahmenbedingungen erfüllt und eine möglichst gute Leistung nach den definierten
Kriterien erzielt. Bei der Grundsätzlichen Architektur des KI handelt es sich um
ein Deep Q-Learning Modell, das auf der Arbeit hinter `Doodle-SDQ'
\cite{zhou_learning_2018} basiert.

Für die Rahmenbedingungen gibt es zwei Ansätze: eine Grundversion und eine
physikalische Version. In der Grundversion kann sich die KI schrittweise um eine
begrenzte Anzahl Pixel auf einer Zeichenfläche fortbewegen. Ausserdem startet
die KI auf einer zufälligen Position auf der Zeichenfläche. Die physikalische
Version ist von simulierter Physik begleitet. So kann die KI durch
Beschleunigungen ihre aktuelle Geschwindigkeit anpassen und sich so fortbewegen,
während diese durch simulierte Reibung kontinuierlich abgebremst wird. 

Für die KI existieren weitere Variationen, die dessen Leistung nach einem
bestimmten Kriterium verbessern sollen. So existiert ein spezifisches Training
auf eine verbesserte Erkennbarkeit und Geschwindigkeit der KI. Durch
Kombinationen der Variationen und der Rahemenbedingungen existieren
schlussendlich acht Versionen der künstlichen Intelligenz.

Die acht Versionen der künstlichen Intelligenz sind alle auf das Nachzeichnen
von Ziffern trainiert. Ein Experiment bestimmt, ob diese Versionen das
Nachzeichnen allgemein erlernen. Die Leistung der Versionen wird auf das
Nachzeichnen von Strichbildern aus dem Quickdraw und dem EMNIST Letters Datenset
gemessen. Wenn die Leistung für diese Strichbilder vergleichbar bleibt mit der
Leistung für die Trainingsdaten, ermöglicht das eine positive Antwort für
die Fragestellung.

Einge Versionen der künstlichen Intelligenz zeigen hierbei vielversprechende
Ergebnisse. Die Grundversion, ohne weitere Variationen, zeichnet in $91\%$ der
Fälle eine erkennbare Ziffer, in $70\%$ der Fälle einen erkennbaren Buchstaben,
und in $72\%$ der Fälle ein erkennbares Motiv aus dem QuickDraw Datenset. 


