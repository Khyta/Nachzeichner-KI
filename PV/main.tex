\documentclass{article}
\usepackage[utf8]{inputenc}
\usepackage[utf8]{inputenc}
\usepackage{german,a4wide,theorem}
\usepackage{graphicx} % for includegraphics
\usepackage{geometry}
\usepackage[hidelinks]{hyperref}
\usepackage{cleveref}
\usepackage{float}
\usepackage[citestyle=numeric, bibstyle=numeric]{biblatex}
\geometry{a4paper,left=20mm,right=20mm, top=3cm, bottom=2.5cm}

\def\frax{\displaystyle\frac} % Change Frac Style

\addbibresource{bib.bib}
\graphicspath{{./}}

\title{\textbf{Nachzeichner KI}}
\author{\textbf{Ian Wasser, Robin Steiner}}
\date{April 2022}

\begin{document}

\maketitle

\tableofcontents

\pagebreak

\section{Thematische Beschreibung}
\label{chap:thematische-beschreibung}
In unserem Projekt versuchen wir folgende Fragestellung zu beantworten: `Kann
eine künstliche Intelligenz lernen, Strichbilder (Bspw. Zahlen, Buchstaben)
nachzuzeichnen, sodass sie durch ein automatisches System erkannt werden
können?'

Das Projekt fällt in den Bereich der Künstlichen Intelligenz, unteranderem weil
menschliche Bewegungen nur schwer algorithmisch aufgefasst werden können.
Deswegen verwenden wir Reinforcement und Deep Learning Ansätze.

Folgende Unterfragen sollen mit dem Projekt beantwortet werden.

\begin{itemize}
    \item Wie kann die Architektur einer KI aussehen, die das Nachzeichnen erlernt?
    \item Wie lässt sich die Leistung der KI in dieser Aufgabe beurteilen?
    \item Wie lässt sich die Leistung von dem Ergebnis verbessern?
    \item Welche Einflüsse haben die Integration von einfachen physischen Rahmenbedingungen auf die Leistungs der KI?
    \item Wie und in wiefern lässt sich die Leistung der KI mit menschlichem Zeichnen vergleichen?
\end{itemize}

\section{Wissensstand, mögliche Quellen}
\label{chap:wissensstand}

Zurzeit kennen wir uns mit den Grundlagen von Machine Learning aus. Wir
verstehen, wie ein Neuronales Netz funktioniert. Allerdings haben wir noch nicht
so viel Erfahrung auf dem Gebiet von Reinforcement. Aus diesem Grund wurde ein
Vorprojekt gemacht, um uns in dieses Gebiet hineinzuarbeiten.

\subsection{Vorprojekt}
Das Vorprojekt stützt besonders auf einer wissenschaftlichen Arbeit aus China.
\url{https://arxiv.org/abs/1810.05977}. Diese Arbeit beschreibt genau die
Architektur einer KI, die das Nachzeichnen erlernt. Unser erstes Unterziel wird
dadurch grösstenteils beantwortet, wodurch wir eine gute Grundlage erarbeiten
konnten, um den Rest des Projekts anzugehen.

Wir haben die Arbeit selbst repliziert und so Erfahrung mit der Technologie und
Hilfsmitteln wie Tensorflow gesammelt.

Die Replikation erzielt zum Zeitpunkt der Projektvereinbarung ansatzweise
ähnliche Ergebnisse wie das Original.


\section{Methodik}
\label{chap:methode}

% Als nächsten Schritt werden wir unser Vorprojekt ausarbeiten und die Leistung
% davon maximieren. Wir orientieren uns dabei weiterhin auf die bereits genannte
% wissenschaftliche Arbeit. Wenn die KI stabile Ergebnisse erzielt, dient diese
% als Grundlage für Experimente und Erweiterungen, die unsere restlichen
% Unterfragen beantworten sollen.

Die Methodik unserer Arbeit ist die iterative Weiterenticklung einer
Computersimulation mit Tensorflow. Recherche wird beigezogen zur Erarbeitung von
Verständnis, als auch für die Lösung von konkreter technischer Probleme.


\subsection{Dokumentation}
\label{chap:m_dokumentation}
Die Dokumentation wird begleitend entwickelt, damit dort die volle Theorie
und unser Vorgehen beschrieben ist. Diese Dokumentation wird auch auf GitHub
verfügbar sein. 


\section{Ressourcen}
\label{chap:ressourcen}

\begin{itemize}
    \item Gute Grafikkarten (GPU accelerated computing) (besitzen wir)
\end{itemize}


\section{Ergebnis}
\label{chap:ergebnis}
Das Ergebnis unseres Projektes soll ein trainiertes Tensorflow-Modell sein,
welches Stichzeichnungen auf möglichst menschliche Weise nachzeichnen kann.


\section{Zeitplan}
\label{chap:zeitplan}
\begin{table}[H]
    \begin{tabular}{ll}
    \textbf{Datum} & \textbf{Beschreibung}                                                                         \\
    03.06.2022     & PV unterschrieben                                                                             \\
    24.06.2022     & Eine KI erschaffen, welche möglichst gut ein Symbol nachzeichnen kann                         \\
    03.07.2022     & Implementierung des Mnist Netzes in die KI                                                    \\
    03.07.2022     & Implementierung der physikalischen Rahmenbedingungen der Simulation                           \\
    07.08.2022     & Optimierung der KI auf physikalische Gegebenheiten                                            \\
    07.08.2022     & Dokumentation, bis auf noch hinzuzufügendene Teile fertig                                     \\
    25.08.2022     & Optimierung der KI auf menschenähnliches zeichnen                                             \\
    \end{tabular}
\end{table}

\section{Bewertungskriterien}
\subsection{Verwendung von Git und GitHub}
\label{chap:git_github}
Wir möchten Git und GitHub verwenden um unser Projekt zu organisieren und zu verwalten. 
Dabei soll bewertet werden, wie konsequent und umfangreich mit diesen Hilfsmitteln gearbeitet wurde?

\subsection{Optimierung der KI}
\label{chap:optimierung}
Wie gross waren die Bemühungen/Erfolge, die Leistung der KI zu verbessern?
Konnte die Leistung der KI verbessert werden?

\subsection{Analyse der KI}
\label{chap:erklaerung}
Wie gut wurden Leistungsunterschiede zwischen einzelnen Versionen der KI evaluiert und analysiert.




\section{Unterschrift}
\label{chap:unterschrift}

Hiermit wird genehmigt, dieses Projekt im Rahmen der Maturarbeit
durchzuführen.

\vspace*{1cm}

Unterschrift: \hrulefill Nicolas Ruh \vspace*{2cm}

Unterschrift: \hrulefill Ian Wasser \vspace*{2cm}

Unterschrift: \hrulefill Robin Steiner \vspace*{2cm}

\printbibliography[heading=bibintoc]

\end{document}